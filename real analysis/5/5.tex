\documentclass{article}
\usepackage{amsmath,amssymb,amsthm}
\usepackage[margin=1 in]{geometry}
%\usepackage{hyperref}
\usepackage{ulem}
\author{2019011985\and M91\and Junzhe Dong}
\title{Homework 5 for Measure and Integral}
\begin{document}
\maketitle
\newcommand{\st}{\text{ s.t.}}
\newcommand{\dx}{\,\mathrm{d} x}
\newcommand{\re}{\mathrm{Re}\,}
\newcommand{\im}{\mathrm{Im}\,}
\newcommand{\sip}{\mathrm{Simp}}
\newcommand{\R}{\mathbf{R}^d}
\newcommand{\oit}{\overline{\int_{\R}}}
\newcommand{\uit}{\uline{\int_{\R}}}
\newcommand{\sit}{\sip\int_{\R}}
\newcommand{\cit}{\int_{\R}}
\newcommand{\WLOG}{without loss of generality}
\newcommand{\M}{\mathcal{M}(\R)}
\newcommand{\uspf}{\sip^+({\R})}
\newcommand{\aeeq}{\stackrel{\mathrm{a.e.}}{=}}


\DeclareRobustCommand{\rchi}{{\mathpalette\irchi\relax}}
\newcommand{\irchi}[2]{\raisebox{\depth}{$#1\chi$}} 

Denote $\M$ the set of Lebesgue measurable sets in $\R$. Denote the linear space of unsigned simple functions $\uspf$


As was proved in previous homework, for an unsigned lebesgue measurable function, $\forall n\in\mathbb{N}, 0\leq k\leq 2^{2n}-1$,define
\begin{equation}\label{folland}
E_n^k=f^{-1}((k2^{-n},(k+1)2^{-n}])\qquad F_n=f^{-1}((2^m,\infty])
\end{equation}
and $\{\phi_n(x)\}\in \uspf$ satisfying
\[\phi_n(x)=\sum_{k=0}^{2^{2n}-1}1_{E_n^k}(x)+2^n1_{F_n}(x)\]
Then $\{\phi_n(x)\}$ is a non-decreasing sequence that converges to $f(x)$, uniformly where $f(x)$ is bounded.

\paragraph{1.3.12}
\begin{proof}
From exercise 1.3.2(iii), we learn that for a Lebesuge measurable set $E$, we have
\[\sip\int_{\R}1_{E}\dx=m(E)\]
$\forall$ union of disjoint boxes $B^k=\bigcup_{n=1}^{\infty}B_i\supset E$, $B$ is Lebesgue measurable with the property that $1_{B^k}\geq 1_{E}$ and that $m^*(B^k)-m(E)\leq \frac 1 k$. By monotonicity
\[\oit 1_{E}\dx\leq\sit 1_{B^k}\dx=m(B^k)\leq m^*(E)+\frac 1 k\]
Take $k\to\infty$ and we get $\leq$ of the equality.

%By definition, $m^*(E)\leq \sum\limits_{n=1}^{\infty}|B_i|$, where $B_i$ are disjoint boxes with the property $\bigcup\limits_{n=1}^{\infty}B_i\subset $

By definition, we may assume that $\forall k\in \mathbf{N},f_k(x)$ is a sequence of unsigned simple functions where \[f_(x)\geq 1_{E}(x)\quad\sit f_n(x)\dx-\oit 1_{E}(x)\dx \leq \frac{1}{k}\]
\WLOG , we assume that $\forall x\in E, f_n(x)=1$ since it's the least value it can take on $E$. By definition and compatibility, $\sit f_n(x)\dx \geq m^*(E)$. So
\[\uit 1_{E}(x)\dx \geq \sit f_n(x)\dx -\frac 1 k\geq m^*(E)-\frac 1 k\]
Take $k\to\infty$ and we get $\geq$ side of the equality.
\end{proof}

\paragraph{1.3.14}
\begin{proof}
By horizontal truncation and vertical truncation, it suffice to show the case where $f$ is unsigned measurable, bounded and finitely supported. By exercise 1.3.11, this indicates that $\uit f(x) \dx=\oit f(x)\dx$.

Now approximate f with \ref{folland}.

Suppose $M$ is a map satisfying the given properties. Take $f(x)=g(x)=0$ into (ii), and we get $M(0+0)=M(0)+M(0)$. So $M(0)=0$.By (ii), $M(f(x))=M(\phi_n(x))+M(f(x)-\phi_n(x))=\sit \phi_n(x)+M(f(x)-\phi_n(x))$. Take $n\to\infty$ and we get $M(f(x))=\uit f(x)\dx +M(0)=\cit f(x)\dx$. This proves the uniqueness.
\end{proof}

\paragraph{1.3.15}
\begin{proof}
Firstly, consider $E\in \M$. Then by translation invariance of Lebesgue measurable sets, $\cit 1_E(x+y)\dx=m(E+y)=m(E)=\cit 1_{E}(x)\dx$. So the statement is true for characteristic functions. 

From linearity of unsigned simple functions, we see that the equality still holds for unsigned simple functions.

Denote F(x)=f(x+y). Then $\forall g(x)\in \uspf$ where $g(x)\leq f(x)\quad\forall x\in \R$, $g(x+y)\leq F(x)$ 
\[\cit f(x)\dx =\uit f(x)\dx= \sup\limits_{\substack{0\leq g\leq f\\g \in\uspf}}\sit g(x)\dx =\sup\limits_{\substack{0\leq g\leq f\\g\in\uspf}}\sit g(x+y)\dx\leq\uit F(x)\dx =\cit F(x)\dx\]
And the $\leq$ side of the inequality is proved.

The other side is similar since $f(x)=f((x+y)-y)$. Thus the equality is proved.
\end{proof}

\paragraph{1.3.16}
\begin{proof}
By horizontal truncation and vertical truncation, it suffice to show the case where $f$ is unsigned measurable, bounded and finitely supported.
%Firstly, $\forall E\in \M$, consider $1_{E}(x)$.
%By taking the usual orthogonal basis $\{e_1,\cdots,e_d\}$, we present $T$ as a matrix $A$: $T(e_1,\cdots,e_d)'=A(e_1,\cdots,e_d)$. 
%$\forall$ boxes $B\in \R$, from compatibility and linear algebra, we learn that $\cit 1_B(x)\dx =$

Firstly, we'll prove that $\forall \text{ boxes } B\in\R$, the equality holds for $1_B(x)$. By translation invariance, we may assume that $T$ fixes the origin. Then by singular value decomposition, it suffice to prove that 
\begin{enumerate}
\item {invariance under rotation and reflection:} Such transformation only translates a disc, so this holds trivially for discs. Meanwhile, a box is the disjoint union of countable discs, so this still holds for boxes.
\item {transformation under stretches:} It suffice to prove the case for one-dimensional cases since for a diagnal matrix $V=diag\{v_1,\cdots,v_d\},\quad\det{V}=\prod\limits_{n=1}^{d}v_n$. 

For one-dimensional case (d=1), $\forall c\in \mathbf{R}_+$ and interval $[a,b]$, we need to prove that 
\[c\cit 1_{[a,b]}(x)\dx=\cit 1_{[a,b]}(\frac 1 c x)\dx\]
A direct verification shows that $1_{[ca,cb]}(x)=1_{[a,b]}(\frac 1 c x)$. So $LHS=c(b-a)=cb-ca=RHS$.
\end{enumerate} 
Now that the equality holds for characteristic functions over boxes, it also holds for that of elementary sets. Since $\forall E\in \M$ where $m(E)<\infty$, $E$ can be approximated by elementary sets, it also holds for that of Lebesgue measurable sets. By exercise 1.3.1(i)(ii), the equality holds for unsigned simple functions.

The rest is similar to the proof in exercise 1.3.15 since $T$ is invertible.
\end{proof}

\paragraph{1.3.18}
\subparagraph{(i)}
\begin{proof}
$\forall n\in\mathbf{N}$, denote $E_n=\{x\in\R:f(x)\geq n\}$. Obviously $E_n\supset E_{n+1}$ and that $E=\bigcap_{n=1}^{\infty}$ is the set where $f$ is not finite. Then it suffice to prove that $m(E)=0$. ($E$ is indeed measurable since $E_n$ are measurable by definition and that $E$ is the countable intersection of them.) 

Denote $\cit f(x)\dx =I<\infty$.

By Markov's inequality, $m(E_n)\leq \frac{1}{n} \cit f(x)\dx <\infty$. So by exercise 1.2.11(ii) (downward monotone convergence), 
\[m(E)=\lim\limits_{n\to\infty}m(E_n)\leq\lim\limits_{n\to\infty}\frac 1 n I=0\]
which proves the statement.
\end{proof}
Counter example: Construct $f(x)$ defined on the Cantor set as follows: Take $f_0(x)=1_{[0,1]}(x)$ and $I_0=[0,1]$. $\forall n\in\mathbf{N}$:
\begin{enumerate}
\item{} Construct $I_{n+1}$ from $I_{n}$ as was done in constructing the usual Cantor set.
\item{} Denfine $f_{n+1}(x)=\frac 4 3 f_n(x)1_{I_{n+1}}(x)$
\end{enumerate}
Then $\forall n\in \mathbf{N}$, $f_n(x)$ is an unsigned simple function where $\sit f_n(x)\dx=1$. Define $f(x)=\lim\limits_{n\to\infty}f_n(x)\dx$, then by definition, $f$ is measurable, infinite on the Cantor set which of zero measure, and $\cit f(x)\dx\geq \sit f_n(x)\dx=1\neq 0$.

\subparagraph{(ii)}
\begin{proof}
The ``if'' side is trivial. 

For the ``only if'' side, suppose the converse is true. Define $E_n=\{x\in\R:f(x)\geq \frac 1 n\}$, then $\exists N\in\mathbf{N}\st m(E_N)\geq 0$. By Markov's inequality, $\cit f(x)\dx\geq \frac 1 N m(E_N)\geq 0$, which is a contrary.
\end{proof}

\paragraph{1.3.19}
\subparagraph{multiplicity by constant}
\begin{proof}
First we assume $f(x)$ is real, where the case $c\geq 0$ is trivial. 
%Let $f(x)=f_+(x)-f_-(x)$, where $f_+(x),f_-(x)$ are unsigned measurable functions, approximated by the non-decreasing sequence of unsigned simple functions $\{f_+^n(x)\}, \{f_-^n(x)\}$ respectively. Then
\[\begin{aligned}
\cit -f(x)\dx=&\cit (-f)_-(x)\dx-\cit(-f)_+(x)\dx\\
=&\cit f_-(x)\dx-\cit f_+(x)\dx\\
=&-\cit f(x)\dx
\end{aligned}\]
Now that we've proved the case where $c=-1$. Notice that $\forall c<0, c=(-1)*|c|$, the equality holds $\forall c\in\mathbf{R}$.
For the general case, denote $c=u+iv$, where $u,v$ are real. Then
\[\begin{aligned}
\cit cf(x)\dx=&\cit(u+iv)(\re f+i\im f)(x)\dx\\
=&\cit (u\re f-v\im f)(x)\dx+i\cit (u\im f+v\re f)(x)\dx\\
=&u\cit \re f\dx-v\cit \re f\dx +iu\cit \im f\dx+iv\cit \re f\dx\\
=&(u+iv)\cit \re f(x)+i\im f(x)\dx\\
=&c\cit f(x)\dx
\end{aligned}\]
\end{proof}

\subparagraph{additivity}
\begin{proof}
First assume that $f(x),g(x)$ are real. Assume that $g(x)\geq 0$. Then
\[\begin{aligned}
\cit f(x)+g(x)\dx=&\cit [f+g]_+(x)\dx-\cit [f+g]_-(x)\dx\\
=&\cit f_+(x)+g(x)\dx-\cit f_-(x)\dx\\
=&\cit f_+(x)\dx+\cit g(x)\dx -\cit f_-(x)\dx\\
=&\cit f(x)\dx+\cit g(x) \dx
\end{aligned}\]
Similarly, the equation is proved for $g(x)\leq 0$ with the help of $-\cit g_-(x)\dx=\cit -g_-(x)\dx$.
For the general real case,
\[\begin{aligned}
\cit f(x)+g(x)\dx=&\cit [f(x)-g_-(x)]+g_+(x)\dx\\
=&\cit f(x)-g_-(x)\dx+\cit g_+(x)\dx\\
=&\cit f(x)\dx+[\cit g_+(x)\dx-\cit g_-(x)\dx\\
=&\cit f(x)\dx+\cit g(x)\dx
\end{aligned}\]
For the general complex case:
\[\begin{aligned}
\cit f(x)+g(x)\dx=&\cit \re (f+g)(x)\dx+i\cit \im (f+g)(x)\dx\\
=&\cit \re f(x)\dx+\cit \re g(x)\dx +i\cit \im f(x)\dx +i\cit \im g(x)\dx\\
=&\cit f(x)\dx+\cit g(x)\dx
\end{aligned}\]
\end{proof}
\subparagraph{conjugate}
\begin{proof}
\[\cit \overline{f(x)}\dx=\cit \re f(x)\dx+i\cit -\im f(x)\dx=\cit \re f(x)\dx-i\cit \im f(x)\dx=\overline{\cit f(x)\dx}\]
\end{proof}

\paragraph{1.3.21}
\subparagraph{``if'' side}
\begin{proof}
Define the sequence $f_k(x)=\sum\limits_{|n|\leq k}c_n1_{[n,n+1)}(x)\quad \forall k\in\mathbf{N}$. Then $\re f_n(x),\im f_n(x)$ are unsigned simple function, and that $\lim\limits_{k\to\infty}f_k(x)=f(x)$.  Therefore, $\re f,\im f$ are measurable functions. Notice that 
\[\cit|\re f| \dx\leq\cit \lim\limits_{k\to\infty}|\re(f_k)|(x)\dx=\lim_{k\to\infty} \cit |\re f_k(x)|\dx=\sum_{n\in\mathbf{Z}}|\re c_n|\leq \sum_{n\in\mathbf{z}}|c_n|<\infty\]
(Taking $\lim$ outside the integral is legitimate since $f_n(x)$ converges uniformly to the bounded function $f$. )
Similarly, $\cit |\im f|\dx<\infty$. So 
\[\cit |f(x)|\dx\leq \cit |\re f(x)|\dx+\cit |\im f(x)|\dx<\infty\]
So $f$ is absolutele integrable.

\[\cit f(x)\dx=\cit \lim_{k\to\infty}f_k(x)\dx =\lim_{k\to\infty}\cit f_k(x)\dx=\lim_{k\to\infty}\sum_{|n|<k}c_n=\sum_{n\in\mathbf{Z}}c_n\]
\end{proof}
\subparagraph{``only if'' side}
\begin{proof}
Denote $S_n=\sum\limits_{|k|<n}c_k$. $|S_n|=\cit |f_n(x)|\dx\leq \cit |f(x)|\dx<\infty$. Take $n\to\infty$ and we learn that $\sum\limits_{n\in\mathbf{Z}}|c_n|=\lim\limits_{n\to\infty}|S_n|<\infty$, so $\sum c_n$ absolutely converges.
\end{proof}
 
\paragraph{1.3.22}
\begin{proof}
$\forall M\in\M$, consider $f(x)=1_M(x)$. Then 
\begin{gather}
\int_{E}f(x)\dx=\cit 1_M(x)1_E(x)\dx=\cit 1_{M\cap E}(x)\dx=m(M\cap E)\\
\int_{E\cup F}f(x)1_E(x)\dx=\int_{E\cup F}1_{M\cap E}(x)\dx=\cit 1_{M\cap E}(x)\dx=m(M\cap E)
\end{gather}
So they coincide. Furthermore, since $E,F$ are disjoint, $M\cap E, M\cap F$ are still disjoint. So
\begin{gather}
\int_{E}f(x)\dx+\int_{F}f(x)\dx=m(E\cap M)+m(F\cap M)=m((E\cup F)\cap M)\\
\int_{E\cup F}f(x)\dx=m((E\cup F)\cap M)
\end{gather}
So they coincide.

Now that we've proved that the equality holds for characteristic functions, by linearity, it still holds for unsigned simple functions. Notice that $\re f_+(x)=\lim\limits_{n\to\infty} \phi_n(x)$ where $\phi_n(x)$ are unsigned simple functions in \ref{folland}. Apply $\phi_n(x)$ to the equalities and take $n\to\infty$ (where taking $\lim$ inside $\int$ is always legitimate since $f$ absolutely converges, thus bounded, $\phi_n(x)$ uniformly converges), and we see that the equalities hold for $\re f_+,\re f_-,\im f_+,\im f_-$. The general equality holds from linearity.
\end{proof}

\paragraph{1.3.23}

\begin{proof}
For a fixed $R>0$, consider $f$ on $B(0,R)\in R$, the ball centered at the origin with radius $R$. Then by hypothesis, $f$ is absolutly integrable on $B(0,R)$
%, so $\forall \varepsilon>0, \exists \{f_n(x)\}$ continuous $\st \| f-f_n\|_{L^1(\R)}\leq \frac {\varepsilon} {4^n}$. 
Similar to the proof on textbook, we learn that $\forall n\in\mathbf{N}, \varepsilon>0, f(x)$ is continuous on $B(0,n)\cap (E_n)^c$, where $m(E_n)<\frac{\varepsilon}{2^n}$. $E=\bigcup\limits_{n=1}^{\infty}E_n$ is the set outside which $f(x)$ is continuous. Meanwhile, by additivity, $m(E)\leq \sum\limits_{n=1}^{\infty}m(E_n)<\varepsilon$, which proves the statement. 

\WLOG, we may replace $f$ by $f1_{|f|\leq n}$ from the horizontal truncation. Thus $f$ is bounded, say $\exists M>0\st|f(x)|<M \forall x\in\R$. Then for arbitraty bounded measurable set $E$, $\int_{E}f(x)\dx\leq M*m(E)<\infty$. Thus $f$ is locally absolutely integrable, which, by the proof for locally absolutely integrable functions, proves the statement.
\end{proof}

\paragraph{1.3.25}
\subparagraph{(i)}
\begin{proof}
Denote $S_n=\int_{B(0,n)}|f(x)|\dx \quad \forall n\in\mathbf{N}$. Then 
\[\lim_{n\to\infty}S_n=\cit |f(x)|\dx<\infty\]
So $S_n$ converges: $\forall \varepsilon>0, \exists N>0, \forall n>N, \cit f(x)\dx-S_n<\varepsilon$. By exercise 1.3.22, this means that $\int_{\R\backslash B(0,n)}f(x)\dx<\varepsilon$, which proves the statement.
\end{proof}
\subparagraph{(ii)}
\begin{proof}
Denote $A_n=\{x\in\mathbf{C}: |x|\leq n\}$. Fix $R>0$, and consider $f^{-1}(A_n)\cap B(0,R)$. Since $f(x)$ is measurable, $f^{-1}(A_n)$ is measurable. Observe that $f^{-1}(A_n)\subset f^{-1}(A_{n+1})$. So
\[\lim\limits_{n\to\infty}m(f^{-1}(A_n)\cap B(0,R))=m(\bigcup\limits_{n=1}^{\infty}f^{-1}(A_n)\cap B(0,R))=m(B(0,R))\]
So $\forall \varepsilon>0,\exists N>0,\st \forall n>N, m(B(0,R))-m(B(0,R)\cap f^{-1}(A_n))<\varepsilon$. From Carath\'eodory criterion, this means $m(B(0,R)\cap (f^{-1}(A_n))^c)<\varepsilon$. The statement follows from the definition of $A_n$.
\end{proof}

\paragraph{1.3.24}

\subparagraph{``if'' side}
\begin{proof}
Claim: A continuous function $f$ is measurable. Since $\forall \text{ open set }O, f^{-1}(O) $ is open, thus measurable, $f$ is indeed measurable. 

Claim: Limit of measurable functions are still measurable. From exercise 1.3.7(iii), it suffice to prove the case for unsigned measurable functions. Suppose $f_n(x)$ is a sequence of measurable functions which a.e. pointwise converges to $f$, then 
\[f^{-1}([a,\infty])= \lim_{n\to\infty}f_n^{-1}([a,\infty])=\overline{\lim_{n\to\infty}}f^{-1}_n([a,\infty])= \bigcup_{n\in\mathbb{N}}f_n^{-1}([a,\infty])\]
So $f^{-1}([a,\infty])\in\M$ , so $f$ is measurable.

By concluding the claims above, ``if'' side is proved. 
\end{proof}

\subparagraph{``only if'' side}
\begin{proof}
From exercise 1.3.7, we learn that $(\re f)_+,(\re f)_-,(\im f)_+,(\im f)_-$ are unsigned measurable simple functions. 

Firstly, we consider $g(x)=(\re f)_+(x)$, which is the pointwise limit of $\phi_n(x)$ in \ref{folland}, where $|g(x)-\phi_n(x)|<\frac 1 {2^n}$ on ponits where $g(x)$ is bounded. By exercise 1.3.25(ii), a measurale function is almost locally bounded. So $\exists \{g_n(x)\}$, a sub-sequence of $\phi_n(x)\st m(\{x\in B(0,n):|g(x)-g_n(x)|\geq \frac 1 {4n}\})<\frac {1}{4^n}$. 
Meanwhile, $g_n(x)=\sum_{i=1}^{N_n}c_i1_{E_i^n}(x)$ where $E_i^n$ are almost disjoint, we have an open set $F_i^n\subset\subset E_i^n \st m^(E_i^n\backslash F_i^n)<\frac{\varepsilon}{2^i}$. Denote $F^n=\bigcup\limits_{i=1}^{N_n}F^n_i $ Thus we get for $R>>1$
\[h_n(x)=\begin{cases}
g_n(x)& \text{if } d(x,F^n)=0\\
c_i*\max\{1-R*d(x,F^n_i),0\}& \text{if } x\in E_i^n\backslash F_i^n\\
0&\text{otherwise}
\end{cases}\]
It's easy to see that $m(\{x\in B(0,n):|g(x)-h_n(x)|\geq \frac 1 {4n}\})<\frac {1}{2^n}$ and that $h(x)$ is continuous.

The statement is similar for the other three functions. By linearity, $\exists \{f_n(x)\}$, a sequence of continuous functions, $\st m(\{x\in B(0,n):|f(x)-f_n(x)|\geq \frac 1 n\})<\frac {1}{2^n}$. Meanwhile, $B=\bigcap_{n=1}^{\infty}\{x\in B(0,n):|f(x)-f_n(x)|\geq \frac 1 n\}$ is the set where $f$ is not approximated by $f_n(x)$ and it is the countable intersection of a decreasing sequence of sets. So by downward monotone convergence, 
\[m(B)=\lim_{n\to\infty}m(\{x\in B(0,n):|f(x)-f_n(x)|\geq \frac 1 n\leq \lim_{n\to\infty}\frac {1}{2^n}=0\]
So $f\aeeq \lim\limits_{n\to\infty}f_n(x)$, where $f_n(x)$ are continuous.
\end{proof}
\subparagraph{proof of Lusin's theorem}
\begin{proof}
%Since continuity is a pointwise property, it suffice to prove the theorem on a finitely measured set.
Now that $f$ is the a.e. pointwise limit of a sequence of continuous functions, by Egorov's theorem, $\forall \varepsilon>0,\exists A\in\M\st m(A)<\varepsilon$ and that $f_n(x)$ converges locally uniformly on $A^c$, so $f(x)$ is continuous on $A^c$ since it's the uniform limit of continuous functions and that continuity is a point-wise property. 
\end{proof}

\end{document}
\ifx\allfiles\undefined
\documentclass{ctexart}
\usepackage{mathrsfs,amsmath,amssymb,amsthm,bm,ulem,comment,hyperref}
\usepackage{tikz-cd}
\usepackage[margin=1 in]{geometry}
\begin{document}

\newcommand{\R}{\mathbb{R}}
\newcommand{\N}{\mathbb{N}}
\newcommand{\dd}{\,\mathrm{d}}
\newcommand{\st}{\text{ s.t. }}
\newcommand{\pp}[2]{\frac{\partial #1}{\partial #2}}
\newcommand{\dif}[2]{\frac{\mathrm{d}#1}{\mathrm{d}#2}}
\newcommand{\nm}[1]{\left\|#1\right\|}
\newcommand{\dual}[1]{\left<#1\right>}
\newcommand{\wto}{\rightharpoonup}
\newcommand{\wsto}{\stackrel{*}{\rightharpoonup}}
\newcommand{\cvin}{\text{ in }}
\newcommand{\alev}{\text{ a.e. }}
\newcommand{\alsu}{\text{ a.s. }}
\newcommand{\E}{\mathcal{E}}
\newcommand{\F}{\mathscr{F}}
\newcommand{\G}{\mathscr{G}}
\newcommand{\Bor}{\mathscr{B}}
\newcommand{\pw}{\text{ p.w. }}
\newcommand{\inof}{\text{ i.o. }}
\newcommand{\X}{\bm{X}}
\newcommand{\iid}{\mathrm{i.i.d.}~}
\newcommand{\C}{\mathbb{C}}

\newtheorem{Thm}{定理}[section]
\newtheorem{Lemma}[Thm]{引理}
\newtheorem{Prop}[Thm]{命题}
\newtheorem{Cor}[Thm]{推论}
\newtheorem{Def}{定义}[section]
\newtheorem{Rmk}{注}[section]
\newtheorem{Eg}{例}[section]
\else
\chapter{可测空间与测度空间}
\fi
\section{集合}
下面以$X$表全集,$\varnothing$表空集,$A\backslash B=\{p\in A:p\not \in B\}$。称$A\backslash B$为真差,若此时$B\subset A$。$A\Delta B=(A\backslash B)\cup(B\backslash A)$称对称差。

若有一族集合$\{A_{t}\in 2^{X}:t\in T\}$。无论$T$基数如何,总可以定义$\bigcap_{t\in T}A_{t}=\{p\in X:\forall t,p\in A_{t}\}$。同理总可以定义$\bigcup_{t\in T}A_{t}=\{p\in X:\exists t\in T, p\in A_{t}\}$

\begin{Thm}[De Morgan]
  \[(\bigcap_{t\in T}A_{t})^{c}=\bigcup_{t\in T}A_{t}^{c}\]
  \[(\bigcup_{t\in T}A_{t})^{c}=\bigcap_{t\in T}A_{t}^{c}\]
  
\end{Thm}
若$A_{n}\nearrow A$,即$\{A_{n}\}$递增,若$A_{n}\subset A_{n+1}$。对于递增的序列,可以定义$A:=\bigcup_{n\in\mathbb{N}}A_{n}=\lim\limits_{n\to\infty}A_{n}$

若$A_{n}\searrow A$,即$\{A_{n}\}$递减,可以定义$A=\bigcap_{n\in\mathbb{N}}A_{n}=\lim\limits_{n\to\infty}A_{n}$

对于一般的集合,不能一般地定义极限。对其定义递减、递增序列:

\[B_{m}=\bigcup_{n=m}^{\infty}A_{n}\quad C_{m}\bigcap_{n=m}^{\infty}A_{n}\]

遂可以定义上、下极限
\[\limsup_{n\to\infty} A_{n}=\lim_{m\to\infty}B_{m}\quad \liminf A_{n}=\lim_{m\to\infty}C_{m}\]称$\lim_{n\to\infty}A_{n}$存在,若$\limsup_{n\to\infty}A_{n}=\liminf_{n\to\infty}A_{n}$。此时$\lim_{n\to\infty}A_{n}=\limsup_{n\to\infty}A_{n}=\liminf_{n\to\infty}A_{n}$

\begin{Prop}
  $A_{n}\nearrow A\Rightarrow F\cap A_{n}\nearrow F\cap A$,$F\backslash A_{n}\searrow F\backslash A$
\end{Prop}

\section{集合系}
\begin{Def}
集合$X$的子集之集$\mathcal{E}$称$X$上的集合系。
\end{Def}

\begin{Def}
  若$\E$是$X$上的集合系,称
  \begin{enumerate}
  \item $\E$是一个$\pi-$系,若$A,B\in \E\Rightarrow A\cap B\in\E$

    即:$\pi-$系对交运算封闭。
  
  \item $\E$是一个半环,若$\E$是一个$\pi-$系,且$\forall A,B\in \E, B\subset A\Rightarrow \exists \{C_{j}\}_{j=1}^{n}\subset \E$两两不交$\st A\backslash B=\bigcup_{j=1}^{n}C_{j}$
   
  \item $\E$是一个环,若$A,B\in\E\Rightarrow A\cup B\in \E,A\backslash B\in \E$。

    一个环一定是半环。

    $A\cap B=A\cup B\backslash (A\Delta B)$,故环同样对交运算封闭。

    环对集合的并和差封闭。此时$\E$在运算$(\cap,\Delta)$的意义下是一个环。(可以视作集合对应的示性函数在$\mathbf{F}_{2}$的$(+,\times)$中构成环)
  \item $\E$是一个代数,若
    \begin{itemize}
    \item $\E$是一个$\pi-$系
    \item $\varnothing\in\E$
      
    \item $\E$对集合的补运算封闭
    \end{itemize}
    故$X\in\E,A\cup B\in \E$,再由De Morgan定律$\E$同样对交封闭。$A\backslash B=A\cap B^{c}$。综上一个代数一定是一个环。
    
  \item $\E$是一个$\sigma$-代数,若$\E$是代数,且其对可数并封闭:$\{C_{n}\}_{n\in\mathbb{N}}\subset \E\Rightarrow\bigcup_{n\in\mathbb{N}}\E$
  \item $\E$是一个单调类,若$A_{n}\searrow A\Rightarrow A\in \E,A_{n}\nearrow A\Rightarrow A\in E$
  \item $\E$是一个$\lambda$系(Dinkin类)若
    \begin{itemize}
    \item $X\in\E$
     
    \item $\E$对真差封闭
      
    \item $\E$对单调序列的极限封闭
    \end{itemize}
%    $X\in\E$,对真差封闭,对单调序列的极限封闭。
  \end{enumerate}
\end{Def}
$\pi-$系$\subset$半环$\subset$环$\subset $代数$\subset \sigma-$代数

单调类$\subset \lambda-$系$\subset\sigma-$代数
\begin{Eg}
  $\R$上左开右闭的集合是半环。%:$A\backslash B=A\backslash (A\cap B)$
\end{Eg}

\begin{Eg}
  $\{\varnothing,X\}$是最小的$\sigma$代数,称平凡$\sigma$代数。
\end{Eg}
\begin{Eg}
  $2^{X}$,即$X$的幂集是最大的$\sigma$代数。
\end{Eg}

\begin{Prop}
  \begin{enumerate}
  \item  设$\E$是一个代数,则它是一个单调类当且仅当$\E$是一个$\sigma$代数。
  \item $\E$是一个$\pi-$系且是$\lambda-$系,则$\E$是$\sigma-$代数。
  \end{enumerate}
 \end{Prop}

 \begin{proof}
   \textbf{1.} 任取$\{A_{n}\}_{n\in\mathbb{N}}\in \E$,只需证$\bigcup_{n\in\mathbb{N}} A_{n}\in \E$。为此定义单调递增序列$B_{n}=\bigcup_{i=1}^{n}A_{i}$。因$\E$是代数,$B_{n}\in E$则$\bigcup_{n\in\N}B_{n}=\bigcup_{n\in \N}A_{n}$

   此时再定义$C_{n}=B_{n}\backslash B_{n-1}$,则$\bigcup_{n\in\N}C_{n}=\bigcup_{n\in\N}A_{n}$
 \end{proof}

 \begin{Rmk}
   今有一族集合系$\{E_{t}\}_{t\in T}$,且$\forall t\in T, \E_{t}$是$X$上的单调类,则$\bigcap_{t\in T}\E_{t}$犹为单调系。
 \end{Rmk}
即:单调类之交犹为单调系。
 
事实上,半环是唯一不满足上述性质的。

\begin{Def}
  设$\E$是$X$上的集合系,称$\mathcal{G}$为$\E$生成的$\sigma-$代数(单调类,$\lambda-$系),若
  \begin{enumerate}
  \item $\E\subset \mathcal{G}$
  \item $\mathcal{G}$是一个$\sigma-$代数(单调类,Dinkin类)
  \item $\forall \mathcal{G}'$满足1.2. 都有$\mathcal{G}\subset\mathcal{G}'$
  \end{enumerate}
  将其记为$\sigma(\E),m(\E),l(\E)$
\end{Def}
$2^{X}$总满足$1,2$,故上述定义的集合总是存在的,且$\sigma(\E)=\bigcap_{t\in T}G_{t}$,其中$t$使得$\forall t\in T, G_{t}$满足1,2。

\begin{Def}
  $X$是一个拓扑空间,$O_{X}=\{X\text{中的开集}\}$,则称$\sigma(O_{X})$为Borel $\sigma-$代数。
\end{Def}

\begin{Prop}
  $Q$是$X$上的一个半环,则其生成的环$r(Q)=Q$中有限无交并$=\bigcup_{n\in\N}\{\bigcup_{j=1}^{n}C_{n,j}: \{C_{n,j}\text{两两不交}\}\}$
\end{Prop}

\begin{Prop}
  \begin{enumerate}
  \item $\E_{1}\subset \E_{2}\Rightarrow m(\E_{1})\subset m(\E_{2}),\sigma(\E_{1})\subset \sigma(\E_{2})$
  \item $\E$是一个代数$\Rightarrow \sigma(\E)=m(\E)$ 
  \item $\E$是一个$\pi-$系$\Rightarrow \sigma(\E)=l(\E)$
  \item $\E_{1}\subset \E_{2}$,$\E_{1}$是代数,$\E_{2}$是单调类$\Rightarrow\sigma(\E_{1})\subset \E_{2}$
  \item $\E_{1}$是$\pi-$系,$\E_{2}$是$\lambda-$系$\Rightarrow \sigma(\E_{1})\subset\E_{2}$
  \end{enumerate}
\end{Prop}

\begin{proof}
  \textbf{1.} trivial

  \textbf{4.} 是2.的推论。

  \textbf{2.} $m(\E)\subset \sigma(\E)$是平凡的。下证明$\sigma(\E)\subset m(\E)$,为此只需证明$m(\E)$是一个代数(因$m(\E)$是一个含$\E$的单调类)。只需证明$X\in m(\E)$,对补封闭,对交封闭。

  \begin{enumerate}
  \item $m(E)$对交封闭:$\forall A,B\in m(\E)$,定义\[G_{1}=\{D\in m(\E):D\cap A\in m(E),\forall A\in m(\E)\}\]\[G_{2}=\{D\in m(\E):D\cap A\in m(\E),\forall A\in E\}\]\[G_{3}=\{D\in m(\E):D^{c}\in m(\E)\}\]目标:证明$m(\E)=G_{1}$,只需证$\E\subset G_{1}$,且$G_{1}$为单调类。

    由于$\E$是代数,易见$\E\subset G_{2}$,且$G_{2}$是单调类:取$\{D_{n}\}\subset G_{2}\st D_{n}\nearrow D\Rightarrow D_{n}\cap A\nearrow D\cap A$,而$D_{n}\cap A\in m(E)$,故得(递减序列是类似的)。故$G_{2}=m(E)$。故$\forall D\in\E,\forall A\in m(\E),D\cap A\in m(\E)$,对换$D,A$得知$G_{1}=m(\E)$。
    \item $m(\E)$对补封闭。$G_{3}$是单调类且$G_{3}\supset \E$,故$G_{3}=m(E)$。
  \end{enumerate}
\end{proof}

\section{集函数}
\begin{Def}
  集函数:集合系$\to \bar{\R}_{+}=[0,+\infty]$
\end{Def}

\begin{Def}[可测空间]
  $(X,\mathscr{F})$,其中$\mathscr{F}$是$X$上的$\sigma$-代数。
\end{Def}

\begin{Def}
  $\nu:\E\to [0,+\infty]$,其中$\E$是$X$上集合系。
  \begin{enumerate}
  \item 单调性:$A,B\in\E,A\subset \E\Rightarrow \nu(A)\leq \nu(B)$
  \item 可减性:$A,B\in \E,A\subset B,B\backslash A\in \E\Rightarrow \nu(B\backslash A)=\nu(A)-\nu(B)$
  \item 有限可加性:$\{E_{i}\}_{i=1}^{n}$两两不交$\Rightarrow \nu(\bigcup_{i=1}^{n}E_{i})=\sum_{i=1}^{n}\nu(E_{i})$
  \item 有限次可加性:$\{E_{i}\}_{i=1}^{n}$不要求两两不交$\Rightarrow \nu(\bigcup_{i=1}^{n}E_{i})\leq\sum_{i=1}^{n}\nu(E_{i})$
 
  \item 可数可加性:$\{E_{i}\}_{i\in\mathbb{N}}$两两不交$\Rightarrow \nu(\bigcup_{i\in\mathbb{N}}E_{i})=\sum_{i\in\mathbb{N}}\nu(E_{i})$
  \item 可数次可加性:$\{E_{i}\}_{i\in\mathbb{N}}$不要求两两不交$\Rightarrow \nu(\bigcup_{i\in\mathbb{N}}E_{i})=\sum_{i\in\mathbb{N}}\nu(E_{i})$
  \item 下连续性:$A_{n}\nearrow A,\{A_{n}\}\subset \E,A\in\E\Rightarrow \nu(\lim_{n\to\infty} A_{n})=\nu(A)=\lim_{n\to\infty}\nu(A_{n})$
  \item 上连续性:$A_{n}\searrow A,\{A_{n}\}\subset \E,A\in\E,\nu(A_{1})<\infty\Rightarrow \nu(\lim_{n\to\infty} A_{n})=\nu(A)=\lim_{n\to\infty}\nu(A_{n})$
  \end{enumerate}
如果$\E$是一个半环,我们称$\nu$为一个预测度,若其满足:
\begin{enumerate}
\item $\nu(\varnothing)=0$
\item 可数可加性
\end{enumerate}
如果$\E$还是一个$\sigma$-代数,则称$\nu$为一个测度,若其满足上述性质。
\end{Def}

\begin{Def}[测度空间]
  可测空间$(X,\mathscr{F})$与其上测度$\nu$构成测度空间。
\end{Def}


若$\nu$是半环$\E$上的预测度,那么上述性质都满足:可数可加$\Rightarrow$有限可加$\Rightarrow$可减性$\Rightarrow$单调性。

\begin{Thm}
  半环$\E$上的预测度$\mu$满足下连续性和可数次可加性。
\end{Thm}


\begin{proof}
  \textbf{下连续性:}设$\{A_{n}\}_{n\in\mathbb{N}}\subset \E,A_{n}\nearrow A,A\in\E$。令$B_{1}=A_{1},B_{n}=A_{n}\backslash (\bigcup_{i=1}^{n}A_{i})\Rightarrow A=\bigcup_{n\in\mathbb{N}}A_{n}=\bigcup_{n\in\mathbb{N}}B_{n}$。

  由半环之定义,$\forall n,\exists\{C_{n,j}\}_{j=1}^{p_{n}}$两两不交$\st B_{n}=\bigcup_{j=1}^{p_{n}}C_{n,j}\Rightarrow A_{m}=\bigcup_{i=1}^{m}\bigcup_{j=1}^{p_{n}}C_{i,j}$。故由有限可加性$\nu(A_{m})=\sum_{i=1}^{m}\sum_{j=1}^{p_{i}}\nu(C_{i,j})$。故  
\[\lim_{m\to\infty}\nu(A_{m})=\lim_{m\to\infty}\sum_{i=1}^{m}\sum_{j=1}^{p_{j}}\nu(C_{i,j})=\sum_{i=1}^{\infty}\sum_{j=1}^{p_{j}}\nu(C_{i,j})=\nu(A)\]
故得。

\textbf{可数次可加性:}设$B_{n}=A_{n}\backslash ((\bigcup_{i=1}^{n}A_{i}))$。因$B\in r(\E)$,其仍然能写成有限不交并:$B_{n}=\bigcup_{j=1}^{n}C_{n,j}\Rightarrow A=\bigcup_{i\in\mathbb{N}}B_{i}=\bigcup_{i=1}^{\infty}\bigcup_{j=1}^{p_{i}}C_{i,j}$

$A_{m}=B_{m}\cup((\bigcup_{i=1}^{m-1}A_{i})\cap A_{m})
%=B_{m}\cup(\bigcup_{i=1}^{m-1}A_{i}\cap A_{m})
%=(\bigcup_{j=1}^{p_{m}}C_{m,j})\cup (\bigcup_{i=1}^{m-1}(A_{i}\cap A_{m}))
=(\bigcup_{j=1}^{p_{m}}C_{m,j})\cup (\bigcup_{i=1}^{m-1}B_{i}\cap A_{m})$
%=(\bigcup_{j=1}^{p_{m}}C_{m,j})\bigcup_{i=1}^{m-1}\bigcup_{j=1}^{p_{i}}C_{i,j}
故
\[\nu(A_{m})=\sum_{j=1}^{p_{m}}\nu(C_{m,j})+\sum_{i=1}^{m-1}\sum_{j=1}^{p_{i}}\nu(A_{m}\cap C_{i,j})
%  \geq \sum_{i=1}^{m}\sum_{j=1}^{p_{i}}\nu(C_{i,j})
\]
求和再由a有限次可加性得$\sum_{m=1}^{M}\nu(A_{m})\geq \sum_{m=1}^{M}\sum_{j=1}^{p_{m}}\nu(C_{m,j})$。令$M\to\infty$,则$\nu(A)\leq \sum_{m=1}^{\infty}\nu(A_{m})$
\end{proof}

\section{Borel-Cantelli引理}
\begin{Thm}
  设$(X,\mathscr{F},\mu)$是测度空间,$\{E_{n}\}\subset\mathscr{F},\sum_{n=1}^{\infty}\nu(E_{n})<\infty\Rightarrow \nu(\limsup\limits_{n\to\infty}E_{n})=0$
\end{Thm}
$\limsup_{n\to\infty}E_{n}=\bigcap_{n\in\mathbb{N}}\bigcup_{i=n}^{\infty}E_{i}$($x\in\limsup_{n\to\infty}\Leftrightarrow x\in E_{n}$ infnitely often)

$\liminf_{n\to\infty} E_{n}=\bigcup_{n\in\mathbb{N}}\bigcap_{i=n}^{\infty} E_{i}$($x\in\liminf_{n\to\infty}E_{n}\Leftrightarrow x\not\in E_{n}$ finitely often)

由此易见$\limsup_{n\to\infty}E_{n}\supset\liminf_{n\to\infty}E_{n}$

\begin{proof}
  由定义:$\limsup_{n\to\infty} E_{n}=\bigcap_{m\in\mathbb{N}}T_{m},T_{m}:=\bigcup_{i=m}^{\infty}E_{i}$。

  $\{T_{m}\}$满足:
  \begin{itemize}
  \item $T_{m}\searrow \limsup_{n\to\infty} E_{n}$
  \item $\nu(T_{1})=\nu(\bigcup_{n=1}^\infty E_{n})\leq \sum_{n=1}^{\infty}\nu(E_{n})<\infty$
  \end{itemize}
  故$\nu(\limsup E_{n})=\lim_{m\to\infty}\nu(T_{m})$。同时$\nu(T_{m})=\nu(\bigcup_{n=m}^{\infty}E_{n})\leq \sum_{n=m}^{\infty}\nu(E_{n})$。令$m\to\infty$,则$\nu(\limsup E_{n})=\nu(T_{m})\to 0$
\end{proof}

\section{测度扩张}

由半环上$Q$的预测度$\nu$,生成测度空间$(X,\mathscr{F},\mu)$
\begin{Eg}
  $Q_{\R}=\{(a,b]:a\leq b\in\R\},\nu((a,b])=F(b)-F(a)$($F$单调增,右连续)是一个预测度。
\end{Eg}
\begin{proof}
显然$\nu(\varnothing)=0$。

\textbf{有限可加性:}设$\{(a_{i},b_{i}]\}_{i=1}^{n}$是$\E$中一族集合。此时只需考虑集合相连的情形。

一方面,若$\bigcup_{i=1}^{n}(a_{i},b_{i}]\subset (c,d]$则由$F$的单调性:$\nu((c,d])=F(d)-F(c)\geq\sum_{i=1}^{n}F(b_{i})-F(a_{i})=\sum_{i=1}^{n}\nu((a_{i,b_{i}}])$。

另一方面,若$(c,d]\subset \bigcup_{i=1}^{n}(a_{i},b_{i}]$则同样由$F$的单调性$ \nu((c,d])\leq \sum_{i=1}^{n}\nu((a_{i},b_{i}])$。

\textbf{可数可加性:}设$\{(a_{i},b_{i}]\}_{i\in\mathbb{N}}$两两不交,$(c,d]=\bigcup_{i\in\mathbb{N}}(a_{i},b_{i}]$。仍然只需考虑集合相连的情形。

一方面,由可数可加性,$\nu((c,d])\geq \sum_{i=1}^{n}\nu((a_{i},b_{i}])$。令$n\to\infty$,得$\nu(c,d]\geq\sum_{i\in\mathbb{N}}\nu((a_{i},b_{i}])$

另一方面,由$F$于$b_{i}$的右连续性,$\forall\varepsilon>0$,$\exists \delta_{i}>0\st 0\leq F(b_{i}+\delta_{i})-F(b_{i})<\frac{\varepsilon}{2^{i+1}}$,故$\{(a_{i},b_{i}+\delta_{i})\}_{i\in\mathbb{N}}$覆盖了$[c+\varepsilon,d]$,由紧性$\exists \{i_{k}:1\leq k\leq K<\infty\}\st [c+\epsilon,d]\subset \bigcup_{i_{k}=1}^{K}(a_{i_{k}},b_{i_{k}}+\delta_{ik}]\Rightarrow$
\[\nu((c+\epsilon,d])\leq \sum_{k=1}^{K}F(b_{i_{k}}+\delta_{i_{k}})-F(a_{i_{k}})<\varepsilon+\sum_{i=1}^{\infty}\nu((a_{i},b_{i}])\]
先令RHS$\varepsilon\to 0$,再令LHS$\epsilon\to 0$,得$\nu((c,d])\leq \sum_{i=1}^{n}\nu((a_{i},b_{i}])$
\end{proof}

\begin{Def}[概率空间]
  称$(X,\mathcal{F},\mu)$ 为概率空间,若$\mathcal{F}$ 是$\sigma$-代数,且$\mu:\mathcal{F}\to [0,+\infty]$满足$\mu(\varnothing)=0$与可数可加性。
\end{Def}
\begin{Eg}[Dirac测度]
  $(\R,2^{\R},\delta),\delta_{0}(A)=\begin{cases}1&0\in A\\0&0\not\in A\end{cases}$
\end{Eg}
\begin{Eg}[Lebesgue测度]
  $(\R,\mathbb{F}_{\lambda},\lambda):$由半环$Q_{\R}=\{(a,b]:a,b\in\R\}$上的预测度$\mu((a,b])=b-a$生成。
\end{Eg}

\begin{Eg}[Lebesgue-Stieltjes]
  $(\R,\mathbb{F}_{\lambda},\lambda_{F}):$由半环$Q_{\R}=\{(a,b]:a,b\in\R\}$上的预测度$\mu((a,b])=F(b)-F(a)$生成,其中$F$单增且右连续。  
\end{Eg}

给定集合$\E$与集函数$\nu$,可以由其扩张为幂集上的外测度$\nu^{*}$:
\begin{Def}[外测度]
  外测度$\tau:2^{X}\to[0,+\infty]$满足
  \begin{itemize}
  \item $\tau(\varnothing)=0$
  \item 单调性:$A\subset B\Rightarrow \tau(A)\leq \tau(B)$
  \item 可数次可加性。
  \end{itemize}
\end{Def}

再从外测度出发限制到子$\sigma$-代数,由Caratheodory定理生成测度。特别地,当$\nu$是预测度时,$(\sigma(\E),\nu^{*}|_{\E})$于上述构造的测度空间只差完备化。

\paragraph{Step 1}
\begin{Prop}
  $(\E,\nu)$ 时$X$上的集合系与$\E$上的非负集函数,且满足$\varnothing\in\E,\nu(\varnothing)=0$。$\forall A\in E$,定义
  \[\nu^{*}(A):=\inf\{\sum_{n\in \N}\nu(E_{n}):\{E_{n}\}_{n\in\N}\subset\E,\bigcup_{n\in\N}E_{n}\supset A\}\]
  特别地,当找不到$A$的覆盖时,定义$\nu^{*}(A)=\infty$。

  这是$X$上的外测度。
\end{Prop}

\begin{proof}
  只需对测度有限的情形证明可数次可加性(无穷的情形是trivial的),即欲证
  \[\nu^{*}(\bigcup_{n\in\N}A_{n})\leq \sum_{n\in\N}\nu^{*}(A_{n})\]
  $\forall n\in \N,\exists \{E_{n,j}\}\subset \E\st \sum_{j\in\N}\nu(E_{n,j})\leq \nu^{*}(A_{n})+\frac{\varepsilon}{2^{n+1}}$ 且$A_{n}\subset \bigcup_{j\in\N}E_{n,j}$。

  对于$A=\bigcup_{n\in \N}A_{n}, \bigcup_{n\in\N}\bigcup_{j\in \N}E_{n,j}\supset A,\nu^{*}(A)<\sum_{n,j\in\N^{2}}\nu(E_{n,j})=\sum_{n\in\N}\nu^{*}(A_{n})+\varepsilon$。令$\varepsilon\to 0$即得。
\end{proof}
\paragraph{Step 2}
\begin{Thm}[Caratheodory]
  若$\tau$是$X$上的外测度,定义
  \[\F_{\tau}:=\{A\subset X:\tau(D)\geq \tau(D\cap A)+\tau(D\cap A^{c})\quad \forall D\subset X\}\]
  则$\F_{\tau}$是完备的测度空间。
\end{Thm}

注意到(由有限次可加性)$\tau(D)\leq \tau(D\cap A)+\tau(D\cap A^{c})$,且对于$\tau(D)=\infty$总有$\tau(D)\geq \tau(D\cap A)+\tau(D\cap A^{c})$,故
\[F_{\tau}:=\{A\subset X:\tau(D)= \tau(D\cap A)+\tau(D\cap A^{c})\quad \forall D\subset X,\tau(D)<\infty\}\]

\begin{Def}[完备性]
  $(X,\F,\mu)$是测度空间。 若$\mu(A)=0$,则称其为零测集;若$B\subset A,\mu(A)=0$,则称其为$\mu$-可略集。记$N$为所有$\mu$-可略集之集合。称该测度空间为$\mu$-完备的,若$N\subset \F$
\end{Def}

\begin{proof}[Caratheodory定理的证明]
  \begin{enumerate}
  \item \textbf{$\bm{\F}$是$\sigma$-代数}

    $X\in\F_{\tau}$,对补对称故封闭,下只需证对可数并封闭。

    对有限并封闭:$\forall A,B\in\F_{\tau}$,我们有如下分解:\[A\cup B=(A\cap B)\cup(A\cap B^{c})\cup (A^{c}\cap B)\]

    故任取$D\subset X\st \tau(D)<\infty$, 有如下无交并之分解
    \[D\cap(A\cup B)=(D\cap A\cap B)\cup (D\cap A\cap B^{c})\cup (D\cap A^{c}\cap B)\]

    故$\tau(D\cap (A\cup B))+\tau(D\cap(A\cup B)^{c})\leq \tau(D\cap A\cap B)+\tau(D\cap A\cap B^{c})+\tau(D\cap A^{c}\cap B)+\tau(D\cap A^{c}\cap B^{c})\leq\tau(D\cap A)+\tau(D\cap A^{c})\leq \tau(D)$

    故$\F_{\tau}$是一个代数。下面证明$\F_{\tau}$是一个$\sigma$-代数(即证明可数可加性)。

    断言:$\forall D\subset X,\tau(D)<\infty,\forall\{E_{i}\}_{i=1}^{n}\subset\F_{\tau}$两两不交,则$\tau(D\cap (\bigcap_{i=1}^{n}E_{i})=\sum_{i=1}^{n}\tau(D\cap E_{i})$。这是因为:
    \begin{align*}
      &\tau(D\cap(\bigcup_{i=1}^{n}E_{i}))\\
      =&\tau(D\cap(\bigcup_{i=1}^{n-1}E_{i})\cap E_{n})+\tau(D\cap(\bigcup_{i=1}^{n-1}E_{i})\cap E_{n}^{c})\\
      =&\tau(D\cap E_{n})+\tau(D\cap (\bigcup_{i=1}^{n-1}E_{i}))\\
      =&\cdots\\
      =&\sum_{i=1}^{n}\tau(D\cap E_{i})
    \end{align*}
    下证$\F_{\tau}$对可数并封闭:$\{E_{i}\}_{i\in\N}\subset \F_{\tau}$且因$\sigma$-代数等价于代数+单调类可不妨假定其两两不交。欲证$\bigcup_{i\in\N}E_{i}\in\F_{\tau}$。$\forall n$
    \begin{align*}
      \tau(D)&=\tau(D\cap (\bigcup_{i=1}^{n}E_{i}))+\tau(D\cap(\bigcup_{i=1}^{n}E_{i})^{c})\\
      &\geq \sum_{i=1}^{n}\tau(D\cap E_{i})+\tau(D\cap(\bigcup_{i\in\N}E_{i})^{c})
    \end{align*}
    令$n\to\infty$即得。
  \item \textbf{$\bm{\tau|_{\F_{\tau}}}$是一个测度}

    可数可加性:$\forall \{E_{i}\}\subset \F_{\tau}$两两不交, $E=\bigcup_{i\in\N}E_{i}$。在上述对$\sigma$-代数证明中$D$取$E$即得$\tau(E)\geq\sum_{i=1}^{\infty}\tau(E_{i})$,故得。

    完备性:$A\in\F_{\tau},\tau(A)=0,B\subset A\Rightarrow \tau(B)=0$. $\forall D\subset X,\tau(D\cap B)+\tau(D\cap B^{c})\leq \tau(B)+\tau(D)=\tau(D)$(各由单调性),完备性得证。
  \end{enumerate}
\end{proof}
\paragraph{Step 3}
问题(测度扩张的存在性):若$(\E,\nu)$是半环及其预测度,由Caratheodory延拓为测度空间$(X,\F_{\nu^{*}},\nu^{*})$,那么$\sigma(\E)\subset\F_{\nu^{*}}$?$\nu^{*}|_{\E}=\nu$?

\begin{Def}[集函数的扩张]
  $\nu:\E\to\infty,\mu:\E_{1}\to[0,+\infty]$。称$\mu$是$\nu$的扩张,若$\E\subset \E_{1}$,且$\mu|_{\E}=\nu$
\end{Def}

\begin{Thm}[Caratheodory-Hahn-Kolmogorov测度扩张定理]
  $(\E,\nu)$是半环及其上的预测度,$(X,\F,\mu)$是其扩张:$\F=\F_{\nu^{*}},\mu=\nu^{*}$,则
  \begin{enumerate}
  \item $\sigma(\E)\subset\F$,即扩张存在。
  \item 若$(X,\sigma(\E),\rho)$为测度空间且$\rho$是$\nu$的扩张,则$\rho(A)\leq\mu(A)\quad\forall A\in\sigma(\E)$,且若$\mu(A)<\infty$,则$\rho(A)=\mu(A)$
  \item 若$\nu$在$\E$上$\sigma$-有限($\exists \{E_{n}\}_{n\in\N}\subset \E,\bigcup_{n\in\N}E_{n}=X$,且$\forall n\in\N,\nu(E_{n})<\infty$),则若$\rho$如上且$\rho=\mu$ on $\E\Rightarrow \rho=\mu$ on $\sigma(\E)$
  \end{enumerate}
\end{Thm}

\begin{proof}
  \begin{enumerate}
  \item 求证$\E\subset \F_{\nu^{*}}$,即证$\forall A\in\E,\forall D\subset X,\nu^{*}(D)\geq \nu^{*}(D\cap A)+\nu^{*}(D\cap A^{c})$.
    \begin{enumerate}
    \item $\forall A,D\in\E\Rightarrow \nu^{*}(D)=\nu^{*}(D\cap A)+\nu^{*}(D\cap A^{c})$
    \item 任取$D\subset X, \nu^{*}(D)<\infty$,求证$\forall\varepsilon>0, \nu^{*}(D)+\varepsilon>\nu^{*}(D\cap A)+\nu^{*}(D\cap A^{c})$。

      首先由Caratheodory延拓之定义,$\exists \{E_{n}\}_{n\in\N}\subset\E,E=\bigcup_{n\in\N}E_{n}\supset D$且$\nu^{*}(D)+\varepsilon>\sum_{n\in\N}\nu(E_{n})$.

      $\forall n\in\N,\nu(E_n)=\nu^{*}(E_{n}\cap A)+\nu^{*}(E_{n}\cap A^{c})$。求和,得$\sum_{n\in\N}\nu(E_{n})=\sum_{n\in\N}\nu^{*}(E_{n}\cap A)+\sum_{n\in\N}\nu^{*}(E_{n}\cap A^{c})\geq\nu^{*}((\bigcup_{n\in\N}E_{n})\cap A)+\nu^{*}((\bigcup_{n\in\N}E_{n})\cap A^{c})\geq \nu^{*}(D\cap A)+\nu^{*}(D\cap A^{c})$
    \end{enumerate}
    
  \item 见讲义
    
  \item 见讲义
  \end{enumerate}
\end{proof}
\paragraph{Step 4}
\begin{Thm}
  若上述3.成立,则$\F_{\nu^{*}}=\overline{\sigma(\E)}$,即$\sigma(\E)$的完备化。
\end{Thm}

\begin{Thm}[完备化定理]
  设$(X,\F,\mu)$是测度空间(不一定完备),$N$为$\mu$-可略集之集合系。定义
  \[\overline{\F}:=\{A\cup B:A\in\F, B\in N\}\quad  \overline{\mu}(A\cup B):=\mu(A)\quad\forall A\in\F, B\in N\]
  则$(X,\overline{F},\overline{\mu})$为一个(良定的)完备的测度空间。
\end{Thm}

\begin{comment}
\begin{Prop}
  $(\E,\nu)$是半环及其上的预测度,则$\nu^{*}|_{\E}=\nu$。
\end{Prop}
\end{comment}

\begin{Eg}
  $(\E,\nu)=(Q_{\R},\nu((a,b])=b-a)$,则由上述扩张便得到了$(\R,\F_{\lambda},\lambda)=\overline{(\R,\mathcal{B}(\R),\lambda)}$是Lebesgue测度。
\end{Eg}

\begin{Eg}
  $X=\mathbb{Q}, \E=\mathbb{Q}\cap Q_{\R},\nu((a,b]\cap \mathbb{Q})=\#((a,b]\cap\mathbb{Q})=
  \begin{cases}
    0& a\geq b\\+\infty&a<b
  \end{cases}  $

  易见$\sigma(\E)=2^{\mathbb{Q}}$,其扩张得到$(\mathbb{Q},\F,\mu),\mu(A)=
  \begin{cases}
    0&A=\varnothing\\+\infty &A\neq\varnothing
  \end{cases}  $

  但$\rho(A)=\#(A)$同样是原测度的扩张。这是因为$\sigma$-有限条件没有得到满足。
\end{Eg}

\section{可测函数及其积分}
%测度空间$(\Omega,\F,P)$称概率空间,若$P(\Omega)=1$。

这是Lebesgue积分的推广,有时仍称Lebesgue积分。
\subsection{可测函数}
\begin{Def}[可测映射]
  $f:X\to Y$, $(X,\F),(Y,M)$是可测空间. $f$称一个$\F-M$可测映射,若$\forall E\in M, f^{-1}(E)\in\F$

  特别地,此时称$f$为可测函数,若$Y=\R,M=\mathcal{B}(\R)$。称$f$为随机变量,若还满足定义域$(\Omega, \F,P)$是概率空间。
\end{Def}
\begin{Rmk}
  $f^{-1}$所指代的不是反函数,而是纤维:$f^{-1}:2^{Y}\to 2^{X},f^{-1}(E)=\{w\in X:f(w)\in E\}$
\end{Rmk}

\begin{Prop}\label{nosig}
  若$M=\sigma(\E),\E\subset Y$,则$f:(X,\F)\to (Y,M)$是$(\F-M)$可测的,当且仅当$f^{-1}(\E)\subset \F$
\end{Prop}
\begin{Prop}
  $\sigma(f^{-1}(\E))=f^{-1}(\sigma(\E))$
\end{Prop}
\begin{proof}
  一方面,$f^{-1}(\sigma(\E))\supset f^{-1}(\E)$。又容易验证$f^{-1}(\sigma(\E))$是一个$\sigma$-代数:
  \[f^{-1}(\bigcup_{\alpha\in T} A_{\alpha})=\bigcup_{\alpha\in T}f^{-1}(A_{\alpha})\]
  且$f^{-1}(Y)=X, f^{-1}(E^{c})=(f^{-1}(E))^{c}$。综上$\sigma(f^{-1}(\E))\subset f^{-1}(\sigma(\E))$

  另一方面,$\sigma(f^{-1}(\E))\supset f^{-1}(\sigma(\E))$。即$\forall E\in\sigma(\E), f^{-1}(E)\in\sigma(f^{-1}(\E))$。定义$G=\{E\in\sigma(\E):f^{-1}(E)\in \sigma(f^{-1}(\E))\}$,只需验证$G$包含$\E$且为$\sigma$-代数。
\end{proof}

\begin{Cor}
  $f:\Omega\to\R, f$可测$\Leftrightarrow\forall a\in\R, \{w:f(w)\in(-\infty,a)\}\in\F$

  (事实上,取区间$(-\infty,a],(a,\infty),[a,\infty)$亦可)
\end{Cor}

即只需验证$\{f<a\}$即可。

\begin{Eg}[指示函数]
  $E\subset X, \bm{1}_{E}(w)=
  \begin{cases}
    1&w\in E\\0 &w\not\in E
  \end{cases}
  $

  \[\{\bm{1}_{E}<a\}=
    \begin{cases}
      \varnothing&a\leq 0\\E^{c} &0<a\leq 1\\X&1<a<\infty
    \end{cases}
  \]
  故欲使$\bm{1}_{E}$可测,只需$E\in\F$
\end{Eg}

\begin{Eg}
  $f:(X,\F)\to (Y,M),g:(Y,M)\to (Z,H)$,$f,g$可测,则$g\circ f:X\to Z$是$\F-H$可测的。
\end{Eg}

\begin{Eg}
  $f:(X,\F)\to (Y,\mathcal{B}),g:(\R,\mathcal{B})\to (\R,\mathcal{B})$,$f,g$可测,则$g\circ f:X\to Z$是$\F-\mathcal{B}$可测的(即Borel可测)。
\end{Eg}

$f: X\to \R,g: \R\to \R$是可测函数$\not\Rightarrow f\circ g$是可测函数:$f$是$(\F-\mathcal{B})$可测,$g$是$\F_{\lambda}-\mathbb{B}$可测:Lebesgue可测集的原像不一定Borel可测。

\begin{Prop}
  $f,g$可测,则其四则运算可测: $f+g, cf,f\cdot g, \frac{f}{g}(g\neq 0)$可测
\end{Prop}

\begin{proof}
  断言:$\{f+g<a\}=\bigcup_{q\in\mathbb{Q}}(\{f<q\}\cap\{g<a-q\})$

  只需证明 LHS$\subset$RHS:若$w$满足$f(w)+g(w)<a$,则$\exists \varepsilon\st f(w)+g(w)<a-\varepsilon$。故$\exists q\in\mathbb{Q},q-\varepsilon<f(w)<q\Rightarrow f(w)<q, g<a-\varepsilon-f<a-q$
\end{proof}

\begin{Rmk}
  $f:\Omega\to\bar\R=\R\cup\{\pm \infty\}, \mathbb{\bar\R}=\sigma(\{[-\infty,a)\}\{[-\infty,a]\},\{[a,+\infty]\},\{(a,+\infty]\})$。$f$可测$\Leftrightarrow$ $\{f\leq a\}\in\F\quad a\in\bar\R$
\end{Rmk}

\begin{Prop}
  $\{f_{n}\}\subset \mathcal{L},\mathcal{L}=\{\text{可测函数}\}$,则$\sup\limits_{n\in\N} f_{n},\inf\limits_{n\in\N}f_{n},\limsup\limits_{n\to\infty}f_{n},\liminf\limits_{n\to\infty}f_{n}$可测
\end{Prop}
\begin{proof}
  $\{\sup\limits_{n\in\N}f_{n}>a\}=\bigcup\limits_{n\in\N}\{f_{n}>a\}\in M$

  $\limsup\limits_{n\to\infty}f_{n}(w)=\inf\limits_{n\in\N}\sup_{m\geq n}f_{m}(w)$
\end{proof}

\begin{Eg}
  存在$E\subset \R$,$E$非Lebesgue可测。

  $\forall \alpha\in(0,1),f_{\alpha}=\bm{1}_{\{\alpha\}}$可测,$f(w)=\sup\{f_{\alpha}(w):\alpha\in E\}=\bm{1}_{E}$非Lebesgue可测。这是因为这不是可数并。
\end{Eg}
\begin{Def}[简单函数 simple function]
  称$f$为简单函数,若$\exists N\in\mathbb{N}\st f(w)=\sum_{i=1}^{N}a_{i}\bm{1}_{E_{i}}(w)$,其中$\{E_{i}\}_{i=1}^{N}\subset \F, \{a_{i}\}_{i=1}^{N}\R$。简单函数都是可测的。$SP=$\{简单函数\},$\mathcal{L}^{+}$=\{非负可测函数\}
\end{Def}

\begin{Prop}[简单函数逼近可测函数]
  \begin{itemize}
  \item $f\in \mathcal{L}^{+},\exists \{f_{n}\}_{n\in\N}\subset SP\cap \mathcal{L}^{+}$,且$f_{n}(w)\nearrow f(w)\quad\forall w\in X$
  \item $f\in \mathcal{L},\exists \{f_{n}\}_{n\in\N}\subset SP,|f_{n}|\nearrow |f|,f_{n}(w)\to f(w)\quad \forall w\in X$
  \end{itemize}
\end{Prop}

\begin{proof}
  $\forall n\in\mathbb{N}$,令$E_{n}=\{f(w)\geq n\}$. $\forall 1\leq k\leq n2^{n}, E_{n,k}=\{\frac{k-1}{2^{n}}\leq f(w)<\frac{k}{2^{n}}\}$,则$\{E_{n},E_{n,k}\}$构成了$X$的一个分割。
  \[f_{n}(w)=n\bm 1_{E_{n}}(w)+\sum_{k=1}^{n2^{n}}\frac{k-1}{2^{n}}\bm{1}_{E_{n,k}}(w)\]
下证明其逐点收敛:$0\leq f(w)-f_{n}(w)\leq
\begin{cases}
  f(w)-n&w\in E_{n}\\ \frac{1}{2^{n}}&w\in E_{n,k}
\end{cases}
$

对非恒正的$f$,只需作分解$f=f^{+}-f^{-}$
\end{proof}

\subsection{积分}
\begin{Def}
  \begin{enumerate}
  \item 若$f\in SP\cap\mathcal{L}^{+},f(w)=\sum_{i=1}^{N}a_{i}\bm{1}_{E_{i}}(w)$, $\int f\dd\mu=\sum_{i=1}^{N}a_{i}\mu(E_{i})$

    注意:简单函数的表示不唯一,取其规范表示:$\sum_{i=1}^{M}b_{i}\bm{1}_{F_{i}}$,其中$b_{i}$两两不同,$\{F_{i}\}$是$X$的分割(两两不同且$\bigcup F_{i}=X$)。规范表示总存在且唯一。事实上,$F_{i}=f^{-1}(b_{i})$。以规范表示代入上述定义便得其良定性。

  \item $\forall f\in\mathcal{L}^{+}=\mathcal{L}(\Omega;[0,+\infty])$,定义$\int_{\Omega}f\dd\mu=\sup\{\int_{\Omega} g\dd \mu:g\in SP(\Omega;[0,+\infty]),g\leq f\}$
  \item $\forall f\in\mathcal{L}$,称其积分存在,若$\int f_{+}\dd\mu,\int f_{-}\dd\mu$存在,并定义$\int f\dd\mu=\int f_{+}\dd\mu-\int f_{-}\dd\mu$
  \end{enumerate}
\end{Def}


\begin{Prop}
  $f\in SP$的积分不依赖于表示方法。
\end{Prop}

\begin{Prop}
  $f,g\in SP\cap \mathbb{L}^{+}$,则
  \begin{enumerate}
  \item $\int cf\dd\mu=c(\int f\dd\mu)\quad\forall c\geq 0$
  \item $\int(f+g)\dd \mu=\int f\dd\mu+\int g\dd\mu$
  \item $f\leq g\Rightarrow \int f\dd\mu\leq\int g\dd\mu$
  \end{enumerate}
\end{Prop}

\begin{proof}
  \textbf{(ii)} 取$f$的规范与不规范表示:$f=\sum_{i=1}^{M}b_{i}\bm 1_{F_{i}}=\sum_{i=1}^{N}a_{i}\bm 1_{E_{i}}$,其中$\{E_{i}\}$是分割但$a_{i}$不要求两两不同(故$M\leq N$),则\textbf{断言}:$\int f\dd\mu=\sum_{i=1}^{N}a_{i}\mu(E_{i})$。

  这是因为:$\forall 1\leq i\leq M,1\leq j\leq N$,则要么$E_{i}\subset F_{j}$,要么$E_{i}\cap F_{j}=\varnothing$。故
  \[RHS=\sum_{i=1}^{N}a_{i}(\sum_{j=1}^{M}\mu(E_{i}\cap F_{j}))=\sum_{j=1}^{M}\sum_{i=1}^{N}a_{i}\mu(E_{i}\cap F_{j})=\sum_{j=1}^{M}\sum_{i=1}^{N}b_{j}\mu(E_{i}\cap F_{j})=\sum_{j=1}^{n}b_{j}\mu(F_{j})=LHS\]

    今取规范表示:$f=\sum_{i=1}^{N}a_{i}\bm{1}_{E_{i}},g=\sum_{k=1}^{K}c_{k}\bm{1}_{G_{k}}$,则$\{G_{k}\},\{E_{i}\}$都是分割,故$\{E_{i}\cap G_{k}\}_{i,k}$仍为分割。$f=\sum_{i=1}^{N}a_{i}(\sum_{k=1}^{K}\bm{1}_{E_{i}\cap G_{k}})=\sum_{i}\sum_{k}a_{i}\bm 1_{E_{i}\cap G_{k}}$。同理,$g=\sum_{i}\sum_{k}c_{k}\bm{1}_{E_{i}\cap G_{k}}$
\end{proof}
\begin{Def}
$X$是概率空间$(\Omega,\F,P)\to\bar\R$上的随机变量,称$E[X]:=\int_{\Omega}X\dd P$为$X$的数学期望。
\end{Def}

\begin{Rmk}
  积分与极限可“交换”:Levi's单调收敛定理,Fatou's引理(对于$\mathcal{L}^{+}$),控制收敛定理(对于$L^{1}$)。
\end{Rmk}

\begin{Prop}
  对于$\mathcal{L}^{+}$,线性性和单调性仍成立。
\end{Prop}
\begin{proof}
对于$f\in \mathcal{L}^{+},c\int f\dd\mu=\int cf\dd\mu$,单调性由定义即得。为证明对加法成立,需用单调收敛定理改写定义。
\end{proof}
\begin{Thm}[单调收敛定理(MCT)]
  $\{f_{n}\}\subset \mathcal L^{+},f\in\mathcal{L}^{+}$,且
  \begin{enumerate}
  \item $f_{n}\leq f_{n+1}$
  \item $f_{n}\to f\quad \text{ p.w. }$ 
  \end{enumerate}
  则
  \[\lim_{n\to\infty}\int f_{n}\dd\mu=\int f\dd\mu\]
\end{Thm}
\begin{proof}
  易见:
  \begin{enumerate}
  \item $f_{n}\leq f_{n+1}\Rightarrow \int f_{n}\dd\mu\nearrow\alpha$
  \item $\forall n\in\N, f_{n}\leq f\Rightarrow \int f_{n}\dd\mu\leq \int f\dd\mu$
  \end{enumerate}
  故$\alpha\leq\int f\dd\mu$。只需证明相反的不等式。为此,只需证明$\forall c\in (0,1),\forall g\in SP(\Omega;\R_{+})$且$g\leq f$,都有
  \[\alpha\geq c\int_{\Omega}g\dd\mu\]

  固定$c$,$g=\sum_{j=1}^{M}b_{j}1_{E_{j}},\{b_{j}\}\subset\R$,则$RHS=\sum_{j=1}^{M}(cb_{j})\mu(E_{j})$。$\forall n\in\N$,定义$F_{n}:=\{cg\leq f_{n}\}$,则$F_{n}\nearrow \Omega$。故由测度的下连续性
  \[RHS=\sum_{j=1}^{M}(cb_{j})\lim_{n\to\infty}\mu(E_{j}\cap F_{n})=\lim_{n\to\infty}\sum_{j=1}^{M}(cb_{j})\mu(E_{j}\cap F_{n})\]
  又$(cg)\bm 1_{F_{n}}=\sum_{j=1}^{M}(cb_{j})\bm 1_{F_{n}}\bm 1_{E_{j}}+ 0\cdot\bm 1_{F_{n}^{c}}$,故
  \[RHS=\lim\limits_{n\to\infty}\int (cg)\bm 1_{F_{n}}\dd\mu\leq \lim\limits_{n\to\infty}\int f_{n}\bm 1_{F_{n}}\dd\mu\leq \int f\dd\mu=\alpha=LHS\]
\end{proof}

\begin{Cor}
  记号如上,则得到非负可测函数积分的另一种定义:
  \[\int f\dd\mu=\lim_{n\to\infty}\int f_{n}\dd\mu=\lim_{n\to\infty}n\mu\{f_{n}\geq n\}+\sum_{k}\frac{k-1}{2^{n}}\mu\{\frac{k-1}{2^{n}}\leq f_{n}<\frac{k}{2^{n}}\}\]
\end{Cor}

由此$\mathcal{L}^{+}$的线性性易得。

\begin{Rmk}
  下述函数不满足单调收敛定理之条件,故极限不可交换:
  \begin{itemize}
  \item $f_{n}=\bm 1_{(n,n+1]}$
  \item $f_{n}=n\bm 1_{(0,\frac{1}{n}]}$
  \end{itemize}
\end{Rmk}

\begin{Thm}[Fatou's引理]
  $\{f_{n}\}\subset \mathcal{L}^{+},f_{n}\to f \quad \pw$,则
  \[\int f\dd\mu=\int(\liminf_{n\to\infty}f_{n})\dd\mu\leq\liminf_{n\to\infty}\int f_{n}\dd\mu\]
\end{Thm}

\begin{proof}
  $\liminf\limits_{n\to\infty} f_{n}(w)=\lim\limits_{n\to\infty}\inf\limits_{m\geq n}f_{m}(w)$。设$g_{n}=\inf\limits_{m\geq n}f_{m}$,则$g_{n}\nearrow \liminf\limits_{n\to\infty} f_{n}$。
  \begin{itemize}
  \item 一方面,由单调收敛定理,$LHS=\int \lim\limits_{n\to\infty}g_{n}\dd\mu=\lim\limits_{n\to\infty}\int g_{n}\dd\mu$。
  \item 另一方面,$g_{n}=\inf\limits_{m\geq n}f_{m}\leq f_{n}(w)\Rightarrow \int g_{n}\dd\mu\leq\int f_{n}\dd\mu\Rightarrow \lim\limits_{n\to\infty}\int g_{n}\dd\mu\leq \liminf\limits_{n\to\infty}\int f_{n}\dd\mu$。
    \end{itemize}
    综上即得。
\end{proof}

\begin{Eg}
  $\forall f\in\mathcal{L}^{+},E\in\F,\nu(E):=\int_{E}f\dd\mu=\int f\bm 1_{E}\dd\mu$是$\F$上的一个测度。
\end{Eg}

\begin{Eg}
  $\{f_{n}\}\subset \mathcal{L}^{+},\sum_{n=1}^{m}f_{n}\nearrow \sum_{n=1}^{\infty}f_{n}\Rightarrow \int \sum_{n=1}^{\infty}f\dd\mu=\sum_{n=1}^{\infty}\int f_{n}\dd\mu$
\end{Eg}

\begin{Eg}
  $f\in\mathcal{L^{+}}\Rightarrow \forall a\in\R^{+},\int f\dd\mu\geq a\mu\{f\geq a\}$
\end{Eg}

\begin{Eg}
  $f\in\mathcal{L^{+}}$,则$\int f\dd\mu=0\Leftrightarrow f=0\quad \alev$
\end{Eg}

\begin{Eg}
  $f,g\in\mathcal{L}^{1},\int |f-g|\dd\mu=0\Leftrightarrow f=g\alev$
\end{Eg}
即零测集上的值不影响积分的值。

\begin{proof}
  $\Leftarrow:$ trivial

  $\Rightarrow:$ $\forall n\in\N,E_{n}=\{f\geq \frac{1}{n}\}\nearrow \Omega$,故$0\geq\frac{1}{n}\mu\{f\geq\frac{1}{n}\}\Rightarrow \mu(f\geq \frac{1}{n})=0$。又$\{f\neq 0\}=\bigcup_{n\in\N}E_{n}$,故得。
\end{proof}

\begin{Cor}
  $\int |f|\dd\mu=0\Leftrightarrow f=0\alev$
\end{Cor}

$f,g$积分存在$\Rightarrow f+g$积分不一定存在。

\begin{Def}
  $\mathcal{L^{1}}=\{f\in\mathcal{L}:\int|f|\dd\mu<\infty\}$
\end{Def}

\begin{Lemma}
  $\forall f,g\in\mathcal{L}^{1}$
  \begin{enumerate}
  \item $\forall c\in\R,cf\in \mathcal{L}^{1},\int_{\Omega}cf\dd\mu=c\int f\dd\mu$
  \item $f+g\in\mathcal{L^{1}}$,且$\int f+g\dd\mu=\int f\dd\mu+\int g\dd\mu$
  \item $f\leq g\Rightarrow \int f\leq \int g$
  \end{enumerate}
\end{Lemma}

\begin{proof}
  $\int |cf|=\int |c||f|=|c|\int |f|<\infty,\int|f+g|\leq \int |f|+\int|g|$。

  对于第一条,只需分别计算$c>0.c<0,c=0$的情形。

  对于第二条,$h_{+}-h_{-}=(f_{+}-f_{-})+(g_{+}-g_{-})\Rightarrow h_{+}+f_{-}+g_{-}=h_{-}+f_{+}+g_{+}$,取积分再由非负函数积分的线性性移项即得。
\end{proof}

$\nm{f}_{\mathcal L^{1}}=\int |f|\dd\mu$不是范数,因为$\nm{f}=0\not\Rightarrow f\equiv 0$

\begin{Def}
  定义等价关系:称$f\sim g$,若$f=g\alev$。定义$L^{1}=\mathcal L^{1}/\sim$。则$L^{1}$犹为线性空间,且$\nm{f}_{L^{1}}$是其上范数。
\end{Def}

\begin{Rmk}
  $f=g\alev$,且$f\in\mathcal{L}$,则$g\in\mathcal{L}$

  $\{f_{n}\}\subset \mathcal{L},f_{n}\to g\alev\Rightarrow g\in \mathcal{L}$

  上述结论对Borel测度不成立(存在Lebesgue零测但Borel非可测的集合)。上述结论成立当且仅当测度空间$(\Omega,\F,\mu)$是完备的。
\end{Rmk}

下文总假定测度空间完备。

\begin{Thm}[控制收敛定理(DCT)]
  $\{f_{n}\}_{n\in\N}\subset L^{1}\st$
  \begin{enumerate}
  \item $f_{n}\to f\alev$
  \item $\exists g\in L^{1}\st |f_{n}|\leq g\alev\quad\forall n\in\N$
  \end{enumerate}
  则$f\in L^{1}$,且\[\lim_{n\to\infty}\int f_{n}\dd\mu=\int f\dd\mu\]
\end{Thm}

\begin{proof}
  易见$f$可测。$|f_{n}|\leq g\Rightarrow |f|\leq g\in L^{1}\Rightarrow f\in L^{1}$。注意到:
  \begin{enumerate}
  \item $g+f_{n}\geq 0\alev$,故由Fatou's 引理
    \begin{align*}
      &\int \lim_{n\to\infty}(g+f_{n})\\
      \leq &\liminf_{n\to\infty}(\int_{\Omega}g+\int_{\Omega}f_{n})\\
      =&\int_{\Omega}g+\liminf_{n\to\infty}\int_{\Omega}f_{n}
      \end{align*}
    两边同消去$\int g$,得$\int_{\Omega}f\leq\liminf\limits_{n\to\infty}\int_{\Omega}f_{n}$
  \item $g-f_{n}\geq 0\alev$。同理可得$\int_{\Omega}f\geq\limsup_{n\to\infty}\int_{\Omega}f_{n}$
  \end{enumerate}
  综上即得。
\end{proof}

\begin{Eg}
  $\{f_{n}\}\subset \mathcal L^{1},\int\sum|f_{n}|<\infty\Rightarrow \int\sum|f_{n}|=\sum\int|f_{n}|$
\end{Eg}
\begin{Eg}
  $SP(\Omega,\R)$在$L^{1}$中稠密。
\end{Eg}

\begin{Eg}[符号测度]
  $f\in L^{1},E\in\F,\nu(E):=\int_{E}f\dd\mu$仍满足$\nu(\varnothing)=0$与可数可加性。
\end{Eg}

\begin{Prop}[Layered公式]
  $X$是随机变量,则\[\sum_{n=1}^{\infty}P\{|X|\geq n\}\leq E[|X|]\leq 1+\sum_{n=1}^{\infty}P\{|X|>n\}\]
\end{Prop}
\begin{proof}
  $|X|=|X|\sum_{n\in\N}\bm 1_{E_{n}},E_{n}=\{n-1\leq|X|< n\}\quad\forall n\in N$,由MCT,$E[|X|]=\sum_{n\in\N}E[|X|\bm 1_{E_{n}}]$。又$(n-1)P(E_{n})\leq E[|X|\bm 1_{E_{n}}]\leq nE[\bm 1_{E_{n}}]=nP(E_{n})$。故
\begin{align*}
  E[|X|]&=\sum_{n=1}^{\infty}\sum_{k=1}^{n}1\cdot P(E_{n})\\
        &=\sum_{k=1}^{\infty}\sum_{n=k}^{\infty}P(E_{n})\\
        &=\sum_{k=1}^{\infty}P(|X|\geq k-1)\\
        &=1+\sum_{k=1}^{\infty}P(|X|\geq k)
\end{align*}
    LHS的证明是类似的。
\end{proof}

\subsection{积分换元公式}
\begin{Thm}
  $(\Omega,\F,\mu)$为测度空间,$(Y,M)$为可测空间,$\phi: \Omega\to Y$为$\F-M$可测的,则$\nu=\phi_{*}\mu: E\mapsto \mu(\phi^{-1}E)$是$M$上的测度,称为$\mu$在$\phi$下的前推(push-forward)。且若$\mu(\Omega)=1$,则$\nu(Y)=1$。

  $f:Y\to\R$为$M$-可测函数,$X:=f\circ \phi$是$\F$可测的,则
  \[\int_{Y}f\dd\nu=\int_{\Omega}f\circ\phi\dd\mu\]
  \[
\begin{tikzcd}                                                                                   & \mathbb{R}_+                                           \\
{(\Omega,\mathscr{F})} \arrow[ru, "\mu", shift right] \arrow[r, "\phi"] \arrow[rd, "f\circ \phi"] & {(Y,M)} \arrow[u, "\mu\circ \phi^{-1}"] \arrow[d, "f"] \\
                                                                                                  & \mathbb{R}                                            
\end{tikzcd}  \]
  
\end{Thm}

\begin{proof}
  用典型方法。

  首先断言:$\forall E\in M$,命题对$f=\bm 1_{E}$成立。

  $LHS=\int_{Y}\bm 1_{E}\dd\nu=\nu(E)=\mu(\phi^{-1}(E))$

  $f\circ \phi=\bm 1_{E}(\phi(w))=\bm 1_{\phi^{-1}(E)}(w)$,故$RHS=\mu(\phi^{-1}(E))$

  再断言上述命题对非负简单函数成立:直接由线性性。再断言对非负可测函数成立:由单调收敛定理。最后将一般的可测函数拆为正部、负部便证明了一般的结论。
\end{proof}

\begin{Eg}
  $X:(\Omega,\F,\mu)\to(\R,\mathscr{B}(\R))$为可测函数,则记$X_{*}\mu=\mu_{X}:\mu_{X}(B):=\mu(\{w\in\Omega:X(w)\in B\})$是$\mathscr{B}(\R)$的测度。若$\mu(\Omega)=1$,则$\mu_{X}$为$(\R,\mathscr{B}(\R))$上的概率测度,称之为$X$诱导的测度(又称$X$的分布(distribution),$X$的law)。
\end{Eg}

\begin{Def}
  $F:\R\to\R$单调增,右连续,则称之为一个分布函数(distribution function)。若还有$\lim\limits_{x\to +\infty}F(x)=1,\lim\limits_{x\to -\infty}F(x)=0$,则称之为一个(累积)概率分布函数(c.d.f.)。

$F_{X}(a):=\mu_{X}\{(-\infty,a]\}$,即由随机变量确定了一个概率分布函数。
\end{Def}

\begin{Rmk}
  给定一个概率分布函数$F$,存在$X:(\Omega,\F,\mu)\to\R\st F_{X}=F$。这样的$X$不唯一。
\end{Rmk}

\begin{Def}
  $F_{X}=F_{Y}$,则称$X=Y\cvin d$,$X$与$Y$同分布(identically distributed)。
\end{Def}

$X=Y\cvin d\not\Rightarrow X=Y$
\begin{Eg}
  $(\Omega,\mathscr{F},P)=([0,1],\F_{\lambda},\lambda)$,考察$X(w)=w.Y(w)=1-w$,则$X=Y\cvin d$,但$X\neq Y\alev$
\end{Eg}

\begin{Thm}
  $f:\R\to\R$ 是Borel(-Borel)可测(即$\forall B\in \mathscr{B}(\R),f^{-1}(B)\in\mathscr{B}(\R)$),则
  \[\int_{\R}f\dd F_{X}=\int_{\R}f\dd\mu_{X} =\int_{\Omega}f(X(\omega))\dd P(w)\]
\end{Thm}

\begin{Eg}
  $E[f(X)]=\int_{\R}f(z)\dd F_{X}(z), E[X]=\int_{\R}x\dd F_{X}(z), E[|X|^{2}]=\int_{R}x^{2}\dd F_{X}$

  $Var[X]=E[|X-E[X]|^{2}]=E[(X-m)^{2}]=E[X^{2}-2mX+m^{2}]=E[|X|^{2}]-(E[X])^{2}$
\end{Eg}

\begin{Eg}
  \begin{enumerate}
  \item purely discrete 离散型:$\mu_{X}$的质量集中在离散点$\{x_{n}\}$上,即$X$的值域为至多可数多个点$\{p_{n}\}$,其中$\mu(\{x_{n}\})=p_{n}>0,\sum_{n}p_{n}=1$。此时$F_{X}$为阶梯函数。此时
    \[E[f(x)]=\int_{\bigcup_{i}\{x_{i}\}}f(x)\dd\mu_{X}=\sum_{i}f(x_{i})\mu_{X}(\{x_{i}\})=\sum_{i}f(x_{i})P(\{X=x_{i}\})=\sum_{i}f(x_{i})p_{i}\]
  \end{enumerate}
\item absolutely continuous 绝对连续型:$\exists f=F'_{X}$,其中$f$非负可积,称$f$为概率密度函数(p.d.f.),$\int_{\R}f\dd\mu=1$,$E[\phi(X)]=\int_{\R}\phi(x)\dd F(x)=\int_{\R}\phi(x)f(x)\dd x$
\end{Eg}

\begin{Eg}
  \begin{enumerate}
  \item Bernoulli: $\mu_{X}(\{0\})=1-p,\mu_{X}(\{1\})=p$
  \item Poisson: $\mu_{X}(\{k\})=\frac{\lambda^{k}}{k!}e^{-\lambda}\quad k\in\N_{+}$
  \item Geometric: $\mu_{X}(\{k\})=p(1-p)^{k-1} k\in\N$
  \end{enumerate}

  \begin{enumerate}
  \item Uniform: $f(x)=\frac{1}{b-a}\bm 1_{[a,b]}$
  \item Normal: $f(x)=\frac{1}{\sqrt{2\pi}}e^{-\frac{x^{2}}{2}},f(x;\mu,\sigma)=\frac{1}{\sqrt{2\pi}\sigma}e^{-\frac{(x-m)^{2}}{2\sigma^{2}}}$
  \item exponential: $f(x)=\lambda e^{-\lambda x}\bm 1_{(0,+\infty)}$
  \end{enumerate}
\end{Eg}

\begin{Eg}
  对于$X\sim P(\lambda)$, $E[X^{2}]=\sum_{k=0}^{\infty}k^{2}\frac{\lambda^{k}}{k!}e^{-\lambda}=\sum_{k=0}^{\infty}k(k-1)\frac{\lambda^{k}}{k!}e^{-\lambda}+\sum_{k=0}^{\infty}k\frac{\lambda^{k}}{k!}e^{-\lambda}=e^{-\lambda}(\lambda^{2}+\lambda)(\sum_{l=0}^{\infty}\frac{\lambda^{l}}{l!})=\lambda^{2}+\lambda$
\end{Eg}

\begin{Def}[矩,moment]
  $p\in \N, E[|X|^{p}]<\infty$,称$E[|X|^{p}]$为$X$的$p$阶矩。
\end{Def}

\subsection{积分不等式}
$\mathscr{L}^{0}:$取无穷的部分为零测集的函数。

\begin{Def}
  $L^{p}(\Omega)=\{X\in \mathscr{L}^{0}:E[|X|^{p}]<\infty\}/\sim$

  $\nm{x}_{p}=(E[|X|^{p}])^{\frac 1 p}\quad 1\leq p<\infty$是$L^{p}(\Omega)$上的一个范数。

  $L^{\infty}(\Omega)=\{X\in\mathscr{L}^{0}:\nm{X}_{\infty}<\infty\}/\sim$

  $\nm{X}_{\infty}=\inf\{a\geq 0:P\{|X|>a\}=0\}=\inf\{a\geq 0: |X|\leq a\quad\alev\}=\sup\{a\geq 0: P\{|X|>a\}>0\}$

  $(L^{\infty},\nm{\cdot}_{\infty})$是一个赋范线性空间。
\end{Def}

\begin{Thm}[H\"older]
  $1\leq p,q\leq \infty,\frac{1}{p}+\frac{1}{q}=1$,则
  \[\nm{XY}_{1}\leq \nm{X}_{p}\nm{Y}_{q}\quad \forall X\in L^{p}, Y\in L^{q}\]
  且对于$1<p,q<\infty$,等号成立$\Leftrightarrow \alpha |X|^{p}=\beta|X|^{q}$对于某$\alpha,\beta\geq 0$成立。
\end{Thm}

\begin{proof}
  利用Young不等式:$ab\leq \frac{1}{p}a^{p}+\frac{1}{q}b^{q}$
\end{proof}

\begin{Rmk}
  $\forall p\in(0,1)$,H\"older不等式不成立。
\end{Rmk}

\begin{Thm}[Minkowski]
  $X,Y\in L^{p}(\Omega)\Rightarrow \nm{X+Y}_{p}\leq \nm{X}_{p}+\nm{Y}_{p}$
\end{Thm}

\begin{Rmk}
  $\forall p\in(0,1)$,Minkowski不等式不成立。
\end{Rmk}

\begin{proof}
  $|X+Y|^{p}\leq (|X|+|Y|)|X+Y|^{p-1}=|X||X+Y|^{p-1}+|Y||X+Y|^{p-1}$,施H\"older不等式即得。
\end{proof}

\begin{Thm}[Jensen]
  设$\phi$凸,$X\in L^{1}(\Omega)$,则$\phi(E[|X|])\leq E[\phi(X)]$
\end{Thm}

\begin{proof}
  $\forall c\in\R,\phi(c)=\sup\{ac+b:ay+b\leq \phi(y)\quad\forall y\}$

  取$c=E[X],\phi(E[X])=\sup\{aE[|X|]+b:ay+b\leq \phi(y)\quad\forall y\}$。若$ay+b\leq\phi(y)$,则$aX(w)+b\leq \phi(X(w))\Rightarrow aE[X]+b\leq E[\phi(X)]$
\end{proof}

\begin{Thm}[Chebyshev]
  $\phi:\R_{+}\to\R_{+}$单增,则$\forall a\in\R_{+}$且$\phi(a)>0$
  \[P\{|X|\geq a\}\leq \frac{E[\phi(X)]}{\phi(a)}\]
\end{Thm}
\begin{proof}
  $E[\phi(|X|)]\geq \int_{\{|X|\geq a\}}\phi(X)\dd\mu\geq \phi(a)\int_{\{|X|\geq a\}}1\dd\mu=\phi(a)P(\{|X|\geq a\})$
\end{proof}

\begin{Eg}[Interpolation]
  $1\leq p\leq q\leq r$,则$L^{p}\cap L^{r}\subset L^{q}$,且
  \[\nm{X}_{q}\leq \nm{X}_{p}^{l}\nm{X}_{r}^{1-l}\quad \frac{1}{q}=\frac{l}{p}+\frac{1-l}{r}\]
\end{Eg}

\begin{Eg}
  $P(\Omega)<\infty, 1\leq p\leq r\leq \infty\Rightarrow L^{r}\subset L^{q}$,且
  \[\nm{X}_{p}\leq \nm{X}_{r}(P(\Omega))^{\frac{1}{q}}\quad \frac{1}{q}+\frac{1}{r}=\frac{1}{p}\]
\end{Eg}

\begin{Rmk}
  上式对$P(\Omega)=\infty$不成立。
\end{Rmk}

\ifx\allfiles\undefined
\end{document}
\fi


%%% Local Variables:
%%% mode: latex
%%% TeX-master: t
%%% End:

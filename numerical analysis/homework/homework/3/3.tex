\documentclass{ctexart}
\usepackage{amsmath,amssymb,amsthm,bm,ulem,graphicx, booktabs}
\usepackage[margin=1 in]{geometry}
\title{数值分析作业3}
\author{数91\and 董浚哲\and 2019011985}
\begin{document}
\maketitle
\newcommand{\R}{\mathbf{R}}
\newcommand{\dd}{\,\mathrm{d}}
\newcommand{\st}{\text{ s.t. }}
\newcommand{\pp}[2]{\frac{\partial #1}{\partial #2}}
\newcommand{\nm}[1]{\left\|#1\right\|}

\paragraph{1.}
注意到\[(I-B)(S^{(k)})-I=-B^{k}\]故
\subparagraph{(1)}
\begin{proof}
若$\rho(B)<1$,则
\[\lim_{k\to\infty}\nm{(I-B)(S^{(k)})-I}=\lim_{k\to\infty}\nm{-B^k}\]
又注意到$\lim\limits_{k\to\infty}\nm{B^k}^{\frac 1 k}=\rho(B)<1$,故$\exists 0<\delta<1,N\in\mathbf{N}\st\forall k>N$
\[\nm{B^k}\leq \delta^k\to 0\quad (k\to\infty)\]

代入上式即得$\lim\limits_{k\to\infty}\nm{(I-B)(S^{(k)})-I}=0$。整理即得$\lim\limits_{k\to\infty}S^{(k)}=(I-B)^{-1}$
\end{proof}
\subparagraph{(2)}
\begin{proof}
反设$\rho(B)\geq 1$。由(i)中结论,知$S^{(k)}\to (I-B)^{-1}\quad(k\to\infty)$,即$\lim\limits_{k\to\infty}\nm{(I-B)(S^{(k)})-I}=\lim\limits_{k\to\infty}\nm{-B^k}=0$。但同时$\nm{-B^k}\geq\rho(B)^k\geq 1$,矛盾!
\end{proof}

\paragraph{2.}
\textbf{Jacobi迭代:}$eig(B_J)=\{0,-\frac{\sqrt{1-a^2}}{2},\frac{\sqrt{1-a^2}}{2}\}$,故迭代收敛$\Leftrightarrow a\in [-1,1]$。

\textbf{Gauss-Seidels 迭代:}$eig(B_G)=\{0,0,\frac{1-a}{4}\}$,故迭代收敛$\Leftrightarrow a\in(-3,5)$。

\paragraph{3.}
\[x^{(k+1)}=(I+\alpha A)x^{(k)}-\alpha b\]
设$B=I+\alpha A$,则$eig(A)=\{\alpha+1,4\alpha+1\}$。故迭代收敛$\Leftrightarrow\rho(B)<1\Leftrightarrow \alpha\in(-\frac{1}{2},0)$。注意到此时$\rho(B)=\begin{cases}\alpha+1&-\frac{2}{5}<\alpha<0\\ -1-4\alpha&-\frac{1}{2}<\alpha<-\frac{2}{5}\end{cases}$,且$R(B)=-\log(\rho(B))$。易知$R(B)$是先增后减的分段单调函数,于$\alpha=-\frac{2}{5}$处取极值,故$\alpha=-\frac{2}{5}$为最佳值。

\paragraph{4.}
\begin{proof}
设$B_J$的特征多项式为$f(\lambda)=\det(\lambda I-B_J)$,则$B_\omega$的特征多项式为$g(\lambda)=\det((\lambda-(1-\omega))I-\omega A)\Rightarrow \omega^nf(\frac{\lambda-(1-\omega)}{\omega})=g(\lambda)$。即若$\lambda_0$是$B_\omega$的特征值,则$\frac{\lambda_0-(1-\omega)}{\omega}$是$B_J$的特征值。特别地,$|\frac{\lambda_0-(1-\omega)}{\omega}|<1\Rightarrow 1-2\omega<\lambda_0<1 $。注意到$0<\omega\leq 1$,故$|\lambda|<1\Rightarrow \rho(B_\omega)<1\Rightarrow$JOR迭代收敛。
\end{proof}

\paragraph{5.}
\begin{proof}
$\forall \mu\in\sigma(B_J),-\mu\in\sigma(B_J),\exists \lambda_\mu^+,\lambda_\mu^-\st (\lambda+\omega-1)^2=\lambda\omega^2\mu^2$。这样的对应是一一对应,故$\sigma(B_J),\sigma(B_\omega)$互相完全确定。

若$\rho(B_J)<1,0<\omega<2$,则$\forall\mu\in\sigma(B_J)$,考察其确定的$\lambda$的取值范围:若$\lambda=0$,则结论显然成立;若$\lambda\neq 0:\mu^2\in[0,1)\Rightarrow 0<\frac{(\lambda+\omega-1)^2}{\lambda}<\omega^2$。由左侧不等式得到$\lambda>0$,解右侧不等式得$(w-1)^2<\lambda<1\Rightarrow |\lambda|<1$,故$\rho(B_\omega)<1$,SOR迭代收敛。

若SOR迭代收敛,由讲义知$0<\omega<2$。若$\lambda=0$,则其对应$\mu=0$,故不妨设$\lambda\neq 0$。
%\[\mu^2=f(\lambda,\omega)=\frac{1}{\lambda}[1+\frac{\lambda-1}{\omega}]^2\quad \omega\in(0,2),\lambda\in (-1,0)\cup (0,1)\]
%,即证$f(\lambda,\omega)<1$。又$\nabla f=\frac{1}{\lambda^2\omega^2}(\omega(1-\omega),\lambda(1-\lambda))$,
%故$f$无驻点,于边界取极值,即$\max|f|<\max\{|f(1,2)|,|f(-1,2)|,|f(1,-2)|,|f(-1,-2)|\}=$

%则需证明在$\lambda\in(-1,1)$的情形下,$\mu^2=\frac{(\lambda+\omega-1)^2}{\lambda\omega^2}<1$。右侧不等式整理得$(\lambda-1)(\lambda-(\omega-1)^2)<0$。因$|\lambda|<1$,故原式等价于证明$\lambda>(1-\omega)^2$。

%求最佳松弛因子即在$\mu$固定的情形下,求使得$\max\lambda$最小的$\omega$。记$f(\lambda)=\lambda^2-(\omega^2\mu^2-2\omega+2)\lambda+(\omega-1)^2$,则其对称轴为$\lambda_c=\frac{\omega^2\mu^2}{2}-(\omega-1)$,$f(\lambda)=0$的两根为$\lambda_\pm=\lambda_c\pm\frac{1}{2}\sqrt{(\omega^2\mu^2)^2-4(\omega-1)\omega^2\mu^2}$。

考察方程
\[\lambda^2-(\omega^2\mu^2-2\omega+2)\lambda+(\omega-1)^2=0\]
其两根为
\[\lambda_\pm(\mu,\omega)=\frac{\omega^2\mu^2}{2}-(\omega-1)\pm\frac{\omega\mu}{2}\sqrt{\omega^2\mu^2-4(\omega-1)}=\frac{1}{4}[\mu\omega\pm\sqrt{\mu^2\omega^2-4\omega+4}]^2>0\]
其中,$\omega^2\mu^2-4\omega+4$的两根为$\omega_0<\omega_1$。由Vieta定理,$\omega_0\omega_1=2$,故$\omega_0<\omega_1$。计算得知$\omega_0=\frac{2-2\sqrt{1-\mu^4}}{\mu^2}=\frac{2}{1+\sqrt{1-\mu^2}}>1$。

为研究$\omega_b$,我们设
\[\rho(B_\omega)=M(\mu,\omega):=\max\{\lambda_+(\mu,\omega),\lambda_-(\mu,\omega)\}\]
注意到其对称性$M(\mu,\omega)=M(-\mu,\omega)$,不妨设$\mu>0$。

\textbf{Case 1:}若$|\mu|\geq 1$,则$\omega^2\mu^2-4(\omega-1)\geq (\omega-2)^2>0$,且$\rho(B_\omega)\geq\lambda_+\geq \frac{\omega^2}{2}-(\omega-1)+\frac{\omega(2-\omega)}{2}=1$,这与SOR迭代收敛矛盾!故$|\mu|<1$

\textbf{Case 2: }若$0<\mu<1$,则

\textbf{Case 2.1:}若$1<\omega_0<\omega<2$,则$\omega^2\mu^2-4(\omega-1)<0$,即$\lambda_\pm$是共轭的两根,由Vieta定理:$M=\sqrt(\lambda_+\lambda_-)=\sqrt{(\omega-1)^2}=\omega-1$

\textbf{Case 2.2:}若$0<\omega<\omega_0$,则$\lambda_+^2-\lambda_-^2=\frac{1}{4}(\lambda_++\lambda_-)\mu\omega\sqrt{\mu^2\omega^2-4(\omega-1)}>0$,故$M(\mu,\omega)=\lambda_+$。

综上所述,$\rho(B_\omega)=\begin{cases}\frac{1}{4}[\omega\mu+\sqrt{\omega^2\mu^2-4(\omega-1)}]^2&\omega\in(0,\omega_0)\\\omega-1&\omega\in [\omega_0,2)\end{cases}$

下面只需要确定$\omega_b\equiv\omega_0$。为此考察$\rho(B_\omega)$的单调性:$\omega>\omega_0$时,严格单调递增;$\omega<\omega_0$时,$\rho(B_\omega)'=\sqrt{\lambda_+}\mu(1-\frac{\omega\mu}{\sqrt{\omega^2\mu^2-4(\omega-1)}})<0$,故单调递减,即$\rho(B_\omega)$于$\omega_0$处取最小值,即$\omega_b\equiv\omega_0$,得证。
\end{proof}


\paragraph{6.}


\begin{tabular}{cccc}
\toprule
& \multicolumn{3}{c}{N=} \\
\cmidrule{2-4}
& 20 & 40 & 80 \\
\midrule
迭代次数 & 42 & 83 & 141 \\
CPU时间 & 0.0809 & 1.5261& 36.8896 \\
\bottomrule
\end{tabular}













\end{document}
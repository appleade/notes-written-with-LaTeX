\documentclass{ctexart}
\usepackage{amsmath,amssymb,amsthm,bm,ulem,graphicx, booktabs}
\usepackage[margin=1 in]{geometry}
\title{数值分析作业7}
\author{数91\and 董浚哲\and 2019011985}
\begin{document}
\maketitle
\newcommand{\R}{\mathbf{R}}
\newcommand{\dd}{\,\mathrm{d}}
\newcommand{\st}{\text{ s.t. }}
\newcommand{\pp}[2]{\frac{\partial #1}{\partial #2}}
\newcommand{\nm}[1]{\left\|#1\right\|}

\paragraph{1.}
\begin{proof}
记$e_i=(\underbrace{0,\cdots,0}_{i-1\text{个}},1,0,\cdots,0)$

剖$Q,M$为列向量:\[Q=[q^k_1,\cdots,q^k_n]\qquad M_k=[m^k_1,\cdots, m^k_n]\]并记$R_k=(r_{ij}^k)$。则条件转化为
\[\lim\limits_{k\to\infty}m^k_l=e_l\quad\forall 1\leq l\leq n\]
所求证转化为
\[\bar{q}_l=\lim\limits_{k\to\infty}q_l^k= e_l\quad\bar r_{ij}=\lim\limits_{k\to\infty}r^k_{ij}= \delta_{ij}\quad \forall 1\leq i<j\leq n,1\leq l\leq n\]

显然$\bar r_{ij}\geq 0$。

注意到如下关系式(由矩阵相乘直接得来)始终成立:
\[m^k_l=\sum_{i=1}^lr^k_{li}q^k_i\quad \forall 1\leq l\leq n\]
用数学归纳法证明结论:
\begin{enumerate}
\item \textbf{Case 1:}$l=1$时,$m_1^k=r_{11}^kq^k_1$。令$k\to\infty$,知$e_1=\bar r_{11}\bar q_1$。注意到$Q$正交,故$\nm{\bar q_1}_2=1$,这只有$q_1=\pm e_1$,但$\bar r_{ij}\geq 0$,这唯有$\bar q_1=e_1,\bar r_{11}=1$。

\item \textbf{Case 2:}设$l<L$时的情形都已得到证明,则$l=L$时:$m_L^k=\sum_{i=1}^Lr_{Li}^kq_i^k$。令$k\to\infty$,得到\[e_L=\sum_{i=1}^{L-1} \bar r_{Li}e_i+r_{LL}\bar q_L\]
由正交性,$e_i^Tq_L=0\quad\forall 1\leq i<L$,且同样由于$r_{LL}\geq 0,\nm{\bar q_L}_2=1$,这唯有$\bar q_{LL}=e_L,r_{LL}=1, r_{iL}=0\quad \forall 1\leq i<L$。
\end{enumerate}
综上得证。
\end{proof}

\paragraph{2.}
\begin{proof}
同讲义,记题干中Rayleigh商为$R(v^{(k)})$。又易见Rayleigh商是伸缩不变的(即$\forall a\in\R\backslash \{0\},\bm{v}\in\R^n,R(a\bm{v})=R(\bm{v})$),故证明该命题等价于证明$R(A^kv^{(0)})=\lambda_1[1+O(|\frac{\lambda_2}{\lambda_1}|^{2k})]$。

由讲义,记$A$的单位正交特征向量为$x_1,\cdots,x_n$,则$A^kv^{(0)}=\lambda_1^k[\alpha_1 x_1+\sum^n_{i=2}(\frac{\lambda_i}{\lambda_1})^k\alpha_i x_i]$。故
\begin{align*}
&R(A^kv^{(0)})\\
=&\frac{\lambda_1^{2k+1}[\alpha_1^2+\sum_{i=2}^n (\frac{\lambda_i}{\lambda_n})^{2k+1}\alpha_i^2]}{\lambda_1^{2k}[\alpha_1^2+\sum_{i=2}^n (\frac{\lambda_i}{\lambda_n})^{2k}\alpha_i^2]}\\
=&\lambda_1[1+\sum_{i=2}^n (\frac{\lambda_i}{\lambda_1})^{2k+1}(\frac{\alpha_i}{\alpha_1})^2][1-\sum_{i=2}^n (\frac{\lambda_i}{\lambda_1})^{2k}(\frac{\alpha_i}{\alpha_1})^2]+o(|\frac{\lambda_2}{\lambda_1}|^{2k})\\
=&\lambda_1[1+O(|\frac{\lambda_2}{\lambda_1}|^{2k})]
\end{align*}
\end{proof}

\paragraph{3.}
$\bar{A}=QAQ^T$,$Q=Q_2Q_1$。其中:(空缺处为0)
\[A=\begin{bmatrix}4&-3& &\\ -3&\frac{38}{9}&\frac{1}{9}&\\ &\frac{1}{9}&\frac{89}{15129}&-\frac{48}{1681}\\&&-\frac{48}{1681}&\frac{63}{1681}\end{bmatrix}\]

\[Q_1=\mathrm{diag}\{1,\frac{1}{3}\begin{bmatrix}-1&2&-2\end{bmatrix}\}\]

\[Q_2=\mathrm{diag}\{I_2,\frac{1}{\sqrt{41}}\begin{bmatrix}5&4\\-4&5\end{bmatrix}\}\]

\paragraph{5.}
\begin{proof}
记$y(\mu)=\nm{Ax-\mu x}^2_2$,则$y'(\mu)=2(\mu (x,x)-(Ax,x)),y''(\mu)=2x^Tx>0$,故该凸函数在极点$\mu=\frac{(Ax,x)}{(x,x)}=R(x)$处取最小值$\nm{Ax-R(x)x}_2$。即
\[\min_{\mu\in\R}\nm{Ax-\mu x}_2^2=\nm{Ax-R(x)x}^2_2\]
两边开平方即得。
\end{proof}


\end{document}